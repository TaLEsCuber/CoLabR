% !TEX root = ../main.tex

% 封面


%---------------------------------------------------------------------
% 评分栏
\begin{table}
	\renewcommand\arraystretch{1.7}
	\begin{tabularx}{\textwidth}{
			|X|X|X|X
			|X|X|X|X|}
		\hline
		\multicolumn{2}{|c|}{Preview Report}&\multicolumn{2}{|c|}{Experimental Record}&\multicolumn{2}{|c|}{Analysis \& Discussion}&\multicolumn{2}{|c|}{\Large\textbf{Total}}\\
		\hline
		\LARGE & & \LARGE & & \LARGE & & \LARGE & \\
		\hline
	\end{tabularx}
\end{table}


%---------------------------------------------------------------------
% 信息栏
\begin{table}
	\renewcommand\arraystretch{1.7}
	\begin{tabularx}{\textwidth}{|X|X|X|X|}
		\hline
		Major \& Grade: & Physics, 2022 &Group number: & Group 2\\
		\hline
		Student name: & \textbf{杨舒云、戴鹏辉、朱政鑫、马万成}  & Student number: & \textbf{22344020、22344016、22344019、22344018}\\
		\hline
		Experiment time: & 2024/10/18 & Teacher's Signature: & \\
		\hline
	\end{tabularx}
\end{table}


%---------------------------------------------------------------------
% 大标题
\begin{center}
	\huge \textbf{APL1-8 \quad 氦氖激光综合实验\\He-Ne Laser Comprehensive Experiment}
\end{center}


%---------------------------------------------------------------------
% 注意事项
\textbf{【Precautions】}
\begin{enumerate}
	\item The lab report consists of three parts:
	\begin{enumerate}
		\item \textbf{Prview Report}: Carefully study the experimental manual before class to understand the experimental principles; familiarize yourself with the instruments, equipment, and tools needed for the experiment, and their usage; complete the pre-lab thought questions; understand the physical quantities to be measured during the experiment, and prepare the experimental record forms in advance as required (you may refer to the experiment report template and print it if needed).
		\item \textbf{Experimental Records}: Meticulously and objectively record the experimental conditions, phenomena observed during the experiment, and data collected. Experimental records should be written in ballpoint pen or fountain pen and signed (\textcolor{fred}{\textbf{Records written in pencil are considered invalid}}). \textcolor{fred}{\textbf{Keep original records, including any errors and deletions; if a correction is necessary due to an error, it must be made according to the standard procedure.}} (Records should not be entered into a computer and printed, but handwritten notes can be scanned and printed); before leaving, have the experimental teacher check and sign the records. 
		\item \textbf{Data Processing and Analysis}: Process the raw experimental data (except for experiments that focus on learning the use of instruments), analyze the reliability and reasonableness of the data; present the data and results in a standardized manner (charts and tables), including numbering and referencing the data, charts, and tables sequentially; analyze the physical phenomena (including answering the experimental thought questions, writing out the thought process, and citing data as needed according to standards); finally, draw a conclusion.
	\end{enumerate}
	\textbf{The experiment report combines the preparation report, experimental records, and data processing and analysis, along with this cover page.}
	\item Submit the \textbf{experiment report} within one week after completing each experiment (under special circumstances, no later than two weeks).
\end{enumerate}


%---------------------------------------------------------------------
% 实验安全	
\textbf{【Safety】}	
\begin{enumerate}
	\item 务必避免激光直射入眼,各小组务必保证本组所用激光不能射出本小组所用实验台。
	\item 请勿随意拆卸光路中已经固定的器件。
	\item 实验过程,请保持站立,请勿落座。
	\item 操作光机组件,请勿大力扭摆,请勿用手接触器件表面。
	\item 使用功率计时注意不要使激光功率超过功率计量程。
	\item 激光器、法布里-珀罗干涉仪、功率计使用和维护参考相应用户手册指导。
	\item 注意一定在相机前插入衰减片,防止强光损坏相机。
\end{enumerate}	
	
	
%---------------------------------------------------------------------	
% 特别鸣谢
\textbf{【Special Note】}	
\begin{itemize}
	\item Special thanks to \textbf{Huanyu Shi}, a senior from the Class of 2019, for providing the \LaTeX \ orginal template for this experiment report, which was adjusted by \textbf{Shuyun Yang}. 
	
	\item \textbf{Due to the absence of an experiment number in the original template, a self-named number has been added for ease of organization on the computer.} 
	
	\item Additionally, \textcolor{fred}{\large\textbf{this experiment report is being improved towards full English expression, so there may be instances of mixed Chinese and English during this transition period. We appreciate your understanding!}}
\end{itemize}

\textbf{【Statement】}
\begin{itemize}
	\LARGE\item 按照老师要求,每个小组提交一份实验报告,因此这篇报告是小组报告,小组成员有\textcolor{fred}{\textbf{杨舒云22344020(本篇报告主要攥写者)、戴鹏辉22344016、朱政鑫22344019和马万成22344018}}。
	
	\item 关于预习报告:由于预习报告由每个同学\textbf{个人独立完成}且在\textbf{实验前就已经提交Seelight系统},因此本篇报告所附预习报告\textbf{仅含有}攥写者本人(杨舒云22344020)的预习报告,其它人的预习报告如果老师想要查看请分别到各人的Seelight中查看。
\end{itemize}


%---------------------------------------------------------------------	
% 目录
\clearpage
\tableofcontents		