% !TEX root = ../main.tex

% 结语部分
\section{差分放大器 \\ The End}


%---------------------------------------------------------------------
% 总结、杂谈与致谢
\subsection{Summary, Thoughts \& Acknowledgments}
\begin{enumerate}
	\item Thank you, teacher, for taking the time to read this experimental report, which still has many shortcomings. I hope you can point out the areas that need improvement. I wish you good health, happiness in life, and success in your work!
	\item Through this experiment, we have learned \textbf{the design, construction and performance testing} of differential amplifiers, and we believe that these knowledge will play a role in our future study or scientific research.
\end{enumerate}


%---------------------------------------------------------------------
% 附件
\subsection{Attachment}
The arrangement of the experimental bench desktop is shown in %\cref{}.

The personal signature of the experimental report is shown in \cref{fig:name1} and \cref{fig:name2}.

\begin{figure}[htbp]
	\centering
	\includegraphics[width=0.5\textwidth]{name.png}
	\caption{signature1}
	\label{fig:name1}
\end{figure}

\begin{figure}[htbp]
	\centering
	\includegraphics[width=0.35\textwidth]{name-TaLEs.jpg}
	\caption{signature2}
	\label{fig:name2}
\end{figure}

\textbf{All relevant code (Python and LaTex source code) has been uploaded to Github.}


%---------------------------------------------------------------------
% 参考文献
\renewcommand{\refname}{Reference}
\begin{thebibliography}{99}	
	\bibitem{a} 维基百科. \emph{维基百科}[M]. https://zh.wikipedia.org
	\bibitem{b} 沈韩. \emph{基础物理实验}[M]. 北京: 科学出版社, 2015.	
\end{thebibliography}