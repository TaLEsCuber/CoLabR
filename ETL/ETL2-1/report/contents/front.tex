% !TEX root = ../main.tex

% 封面


%---------------------------------------------------------------------
% 评分栏
\begin{table}
	\renewcommand\arraystretch{1.7}
	\begin{tabularx}{\textwidth}{
			|X|X|X|X
			|X|X|X|X|}
		\hline
		\multicolumn{2}{|c|}{Preview Report}&\multicolumn{2}{|c|}{Experimental Record}&\multicolumn{2}{|c|}{Analysis \& Discussion}&\multicolumn{2}{|c|}{\Large\textbf{Total}}\\
		\hline
		\LARGE & & \LARGE & & \LARGE & & \LARGE & \\
		\hline
	\end{tabularx}
\end{table}


%---------------------------------------------------------------------
% 信息栏
\begin{table}
	\renewcommand\arraystretch{1.7}
	\begin{tabularx}{\textwidth}{|X|X|X|X|}
		\hline
		Grade \& Major: & 2022, Physics &Group number: & A2\\
		\hline
		Student name: & 杨舒云 \& 戴鹏辉  & Student number: & 22344020 \& 22344016\\
		\hline
		Experiment time: & 2024/9/25 & Teacher's Signature: & \\
		\hline
	\end{tabularx}
\end{table}


%---------------------------------------------------------------------
% 大标题
\begin{center}
	\huge \textbf{ET2-1 \quad 蓝牙音箱的焊接和调试\\Welding and Debugging of Bluetooth Speakers}
\end{center}


%---------------------------------------------------------------------
% 注意事项
\textbf{【Precautions】}
\begin{enumerate}
	\item The lab report consists of three parts:
	\begin{enumerate}
		\item \textbf{Prview Report}: Carefully study the experimental manual before class to understand the experimental principles; familiarize yourself with the instruments, equipment, and tools needed for the experiment, and their usage; complete the pre-lab thought questions; understand the physical quantities to be measured during the experiment, and prepare the experimental record forms in advance as required (you may refer to the experiment report template and print it if needed).
		\item \textbf{Experimental Records}: Meticulously and objectively record the experimental conditions, phenomena observed during the experiment, and data collected. Experimental records should be written in ballpoint pen or fountain pen and signed (\textcolor{fred}{\textbf{Records written in pencil are considered invalid}}). \textcolor{fred}{\textbf{Keep original records, including any errors and deletions; if a correction is necessary due to an error, it must be made according to the standard procedure.}} (Records should not be entered into a computer and printed, but handwritten notes can be scanned and printed); before leaving, have the experimental teacher check and sign the records. 
		\item \textbf{Data Processing and Analysis}: Process the raw experimental data (except for experiments that focus on learning the use of instruments), analyze the reliability and reasonableness of the data; present the data and results in a standardized manner (charts and tables), including numbering and referencing the data, charts, and tables sequentially; analyze the physical phenomena (including answering the experimental thought questions, writing out the thought process, and citing data as needed according to standards); finally, draw a conclusion.
	\end{enumerate}
	\textbf{The experiment report combines the preparation report, experimental records, and data processing and analysis, along with this cover page.}
	\item Submit the \textbf{experiment report} within one week after completing each experiment (under special circumstances, no later than two weeks).
\end{enumerate}


%---------------------------------------------------------------------
% 实验安全	
%\textbf{【Safety】}	
%\begin{enumerate}
%	\item 
%\end{enumerate}	
	
	
%---------------------------------------------------------------------	
% 特别鸣谢
\textbf{【Special Note】}	

Special thanks to \textbf{Huanyu Shi}, a senior from the Class of 2019, for providing the \LaTeX \ template for this experiment report. 

\textbf{Due to the absence of an experiment number in the original template, a self-named number has been added for ease of organization on the computer.} 

Additionally, \large\textbf{\textcolor{fred}{this experiment report is being improved towards full English expression, so there may be instances of mixed Chinese and English during this transition period. We appreciate your understanding!}}
	

%---------------------------------------------------------------------	
% 目录
\clearpage
\tableofcontents		