% !TEX root = ../main.tex

% 实验记录	
\begin{table}
	\renewcommand\arraystretch{1.7}
	\centering
	\begin{tabularx}{\textwidth}{|X|X|X|X|}
		\hline
		Major: & Physics & Grade: & 2022 \\
		\hline
		Name: & 杨舒云 & Student number: & 22344020\\
		\hline
		Room temperature: & \degree C & Experimental location: &  \\
		\hline
		Student's Signature:& \textbf{In Attachment} & Score: &\\
		\hline
		Experiment time:& 2024// & Teacher's Signature:&\\
		\hline
	\end{tabularx}
\end{table}
\section{XXX \\ Experimental Record}


%---------------------------------------------------------------------
% 实验过程记录
\subsection{Content, Procedures \& Results}

% 操作步骤
\subsubsection{Operations}
\begin{enumerate}
	\item 
\end{enumerate}	

% 实验结果
\subsubsection{Display}

The results are shown in \cref{tab:tab1}.

\begin{enumerate}
	\item \begin{table}[h]
		\centering
		\caption{Examples of table}
		\label{tab:tab1}
		\begin{tabular}{|c|c|c|c|c|c|}
			\hline
			组1/序号i & 1 & 2 & 3 & 4 & 5 \\
			$v_{1i}(m/s)$ & 1.26 & 1.08 & 1.00 & 0.75 & 0.38 \\
			$f_{1i}(Hz)$ & 40073 & 40127 & 40105 & 40088 & 40066 \\
			\hline
			组2/序号i & 1 & 2 & 3 & 4 & 5 \\
			$v_{2i}(m/s)$ & 1.21 & 1.06 & 0.99 & 0.52 & 0.57 \\
			$f_{2i}(Hz)$ & 40143 & 40125 & 40084 & 40080 & 40067 \\
			\hline
			组3/序号i & 1 & 2 & 3 & 4 & 5 \\
			$v_{3i}(m/s)$ & 1.15 & 0.98 & 0.78 & 0.59 & 0.36 \\
			$f_{3i}(Hz)$ & 40135 & 40115 & 40092 & 40070 & 40044 \\
			\hline
		\end{tabular}
	\end{table}		
\end{enumerate}


%---------------------------------------------------------------------
% 原始数据
\subsection{Original Data}
The original data in the experimental notebook is shown in %\cref{} (signed).

See the \textbf{Attachment} section for the clean of the experimental bench desktop (%\cref{}).

Other raw data are shown in %\cref{}.


%---------------------------------------------------------------------
% 问题记录
\subsection{Difficulties}
\begin{enumerate}
	\item 
\end{enumerate}
% ---