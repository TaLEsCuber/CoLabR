\documentclass[12pt]{article}
\usepackage{graphicx}
\usepackage{geometry}
\usepackage{amsmath}
\usepackage{cite}

\input{Settings} % 导入模板的相关设置


\geometry{a4paper, margin=1in}

\title{热力学第二定律设计性实验}
\author{
    姓名:\underline{\hspace{3cm}} \\
    学号:\underline{\hspace{3cm}} \\
    姓名:\underline{\hspace{3cm}} \\
    学号:\underline{\hspace{3cm}}
}
\date{\today}

\begin{document}

\maketitle

\section*{摘要}
(200字左右,简炼总结实验方案特点和达到的指标,实验结果等)

\begin{quote}
\textit{在此处写入摘要内容……}
\end{quote}

\section*{背景介绍}
(热力学第二定律背景介绍,热机背景知识了解,引参考文献,1-1.5页)

\subsection*{热力学第二定律}
在此处写入热力学第二定律的背景介绍……

\subsection*{克劳修斯表述}
在此处写入克劳修斯表述的内容……

\section*{基本原理与实验方案}
(介绍设计的原理、方案构想、测量方程等。基于测量方法讨论误差分配,评估可行性。带有原理图、公式等,1-2页)

在此处介绍基本原理与实验方案……

\section*{实验搭建}
(介绍实验装置,实物图展示(可多个图),实验用仪器等,1-2页)

在此处介绍实验装置和实验用仪器……

% \begin{figure}[h]
%     \centering
%     \includegraphics[width=0.8\textwidth]{example.jpg} % 将example.jpg替换为您的实验装置图片文件名
%     \caption{实验装置图}
%     \label{fig:experiment_setup}
% \end{figure}

\section*{数据分析、误差讨论}
(介绍数据来源、意义以及处理方法,画图并进行误差分析,2页以内)

在此处介绍数据分析和误差讨论……

\section*{总结及存在问题}
(简要总结实验结论和存在问题,后续可行的改进措施等)

在此处写入总结及存在问题……

\bibliographystyle{plain}
% \bibliography{references} % 将references替换为您的参考文献文件名

\end{document}
