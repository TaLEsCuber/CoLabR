% 说明:基于2019级大佬石寰宇的模板进行魔改,详见https://zhuanlan.zhihu.com/p/620722147;在此特别感谢大佬的工作!
% 魔改参考了:https://www.zhihu.com/question/29676847
% 说明:本设定用于Beamer制作。



% ---------------------------------------------------------------------
% 自定义颜色
% Firefly配色!!!
\usepackage{xcolor}
\definecolor{c1}{HTML}{3E324A} % #3e324a(紫黑)
\definecolor{c2}{HTML}{475D7B} % #475d7b(灰蓝)
\definecolor{c3}{HTML}{97C6C0} % #97c6c0(灰绿)
\definecolor{c4}{HTML}{E26E1B} % #e26e1b(深橘红)
\definecolor{c5}{HTML}{E6E4E0} % #e6e4e0(银白)
\definecolor{c6}{HTML}{4DF8E8} % #4df8e8(蓝绿)
\definecolor{c7}{HTML}{C2D5CD} % #c2d5cd(茉绿)


% ---------------------------------------------------------------------
% 设置主题颜色

% 方式1
% 用这一个:\documentclass{beamer}\usetheme{Madrid}
%\usecolortheme[named=c1]{structure}
%\setbeamercolor{frametitle}{fg=c2, bg=c6}
%\setbeamercolor{item}{fg=c3}
%\setbeamercolor{item projected}{fg=c1}
%\setbeamercolor{title}{fg=c2, bg=c6}


% 方式2
% 采用方式二就不要用主题;下面魔改了一个空白的主题
% 用这一个:\documentclass[aspectratio=169]{beamer}
% 加导航条
\useoutertheme[width=3\baselineskip,right]{sidebar}
%\useoutertheme[width=3\baselineskip,left]{sidebar}
\setbeamercolor{section in sidebar}{fg=c2, bg=c3}
%\setbeamercolor{section in sidebar}{fg=c2, bg=c6}
%\setbeamerfont{section in sidebar}{series=\bfseries}
% 目录标数字
\setbeamertemplate{section in toc}[sections numbered] 
%\setbeamertemplate{subsection in toc}[subsections numbered]
%\setbeamertemplate{subsubsection in toc}[subsubsections numbered]
% 无序列表用实心点
\setbeamertemplate{itemize item}{$\bullet$}
\setbeamertemplate{enumerate item}{\textbf{\insertenumlabel}.}
% 去掉下面没用的导航条
\setbeamertemplate{navigation symbols}{}
% 设置页脚格式
\makeatother
\setbeamertemplate{footline}
{
	\leavevmode%
	\hbox{%
		\begin{beamercolorbox}[wd=.4\paperwidth,ht=2.25ex,dp=1ex,center]{author in head/foot}%
			\usebeamerfont{author in head/foot}\insertauthor
		\end{beamercolorbox}
		
		\begin{beamercolorbox}[wd=.6\paperwidth,ht=2.25ex,dp=1ex,center]{title in head/foot}%
			\usebeamerfont{title in head/foot}\inserttitle\hspace*{13em}
			\insertframenumber{} / \inserttotalframenumber\hspace*{0ex}
	\end{beamercolorbox}}
	
	\vskip0pt%
}
\makeatletter
% 不同元素指定不同颜色,fg是本身颜色,bg是背景颜色,!num!改变数值提供渐变色
\setbeamercolor{title}{fg=c2, bg=c5}
\setbeamercolor{frametitle}{fg=c2, bg=c5}
% 设置页脚对应位置颜色
\setbeamercolor{author in head/foot}{fg=c5, bg=c1}
\setbeamercolor{title in head/foot}{fg=c5, bg=c1}
% 设置sidebar颜色
%\setbeamercolor{sidebar right}{bg=c7}
\setbeamercolor{sidebar right}{bg=c5}
%\setbeamercolor{sidebar left}{bg=c7}
\setbeamercolor{structure}{fg=c2, bg=c3} 
\setbeamercolor{item}{fg=c6}
% 左右页间距的排版
\def\swidth{2.3cm}
%\setbeamersize{sidebar width right=\swidth}
%\setbeamersize{sidebar width left=\swidth}
\setbeamersize{sidebar width right=1.8cm}
\setbeamersize{sidebar width left=1.8cm}
\setbeamerfont{title in sidebar}{size=\scriptsize}
\setbeamerfont{section in sidebar}{size=\tiny}


% ---------------------------------------------------------------------
% 引入必要的包
\usepackage{amsmath,enumerate,multirow,float}
\usepackage{tabularx}
\usepackage{tabu}
\usepackage{subfig}
\usepackage{fancyhdr}
\usepackage{graphicx}
\usepackage{wrapfig}  
\usepackage{physics}
\usepackage{appendix}
\usepackage{amsfonts}

\usepackage{mathrsfs} % 字体
\usepackage{calligra}
\usepackage{lipsum}
\usepackage{adjustbox}
\usepackage{tabularray} % 绘制表格时可以更加方便添加框线

% ---------------------------------------------------------------------
% 颜色盒子
\usepackage{tcolorbox}
\tcbuselibrary{skins,breakable}

\newtcolorbox[auto counter,number within=section]{myhighlight}[1][]{
  top=2pt,bottom=2pt,arc=1mm,
  boxrule=0.5pt,
%   frame hidden,
  breakable,
  enhanced, %跨页后不会显示下边框
  coltitle=c2!80!gray,
  colframe=c3,
  colback=c6,
  drop fuzzy shadow,
  title={Key Point~\thetcbcounter:\quad},
  fonttitle=\bfseries,
  attach title to upper,
  #1
}



% ---------------------------------------------------------------------
% 利用cleveref改变引用格式,\cref是引用命令
\usepackage{cleveref}
\crefformat{figure}{#2{\textcolor{c2}{\textbf{Figure #1}}}#3} % 图片的引用格式
\crefformat{equation}{#2{(\textcolor{c2}{#1})}#3} % 公式的引用格式
\crefformat{table}{#2{\textcolor{c2}{\textbf{Table #1}}}#3} % 表格的引用格式



% ---------------------------------------------------------------------
%	对目录、章节标题的设置
%\usepackage{hyperref} 
%\hypersetup{
%	colorlinks,
%	linktoc = section, % 超链接位置,选项有section, page, all
%	linkcolor = c2, % linkcolor 目录颜色
%	citecolor = c2  % citecolor 引用颜色
%}


% ---------------------------------------------------------------------
%   listing代码环境设置
\usepackage{listings}
\lstloadlanguages{python}
\lstdefinestyle{pythonstyle}{
backgroundcolor=\color{gray!5},
language=python,
frameround=tftt,
frame=shadowbox, 
keepspaces=true,
breaklines,
columns=spaceflexible,                   
basicstyle=\ttfamily\small, % 基本文本设置,字体为teletype,大小为scriptsize
keywordstyle=[1]\color{c1}\bfseries, 
keywordstyle=[2]\color{Red!70!black},   
stringstyle=\color{Purple},       
showstringspaces=false,
commentstyle=\ttfamily\scriptsize\color{green!40!black},%注释文本设置,字体为sf,大小为smaller
tabsize=2,
morekeywords={as},
morekeywords=[2]{np, plt, sp},
numbers=left, % 代码行数
numberstyle=\it\tiny\color{gray}, % 代码行数的数字字体设置
stepnumber=1,
rulesepcolor=\color{gray!30!white}
}



% ---------------------------------------------------------------------
%	其他设置
\def\degree{${}^{\circ}$} % 角度
\graphicspath{{./images/}} % 插入图片的相对路径
\allowdisplaybreaks[4]  %允许公式跨页