% ----------------------------------------------------- 
%	说明
% ----------------------------------------------------- 

% 本模板基于YSY的优秀实验报告模板进行优化,其完成了绝大部分工作,在此表示衷心感谢其支持与贡献。
% 无论您通过何种途径获得本文件,务必确保文件内容的隐私得到充分保护。请注意,本文件不支持上传任何平台,若因文件内容引发任何权益纠纷,文件的基础参与建设者将不承担任何责任。
% 若您愿意与我们进行进一步交流,欢迎通过以下网址与我们联系:https:// 。我们将在与相关人员沟通后,为您提供进一步的技术支持,我们相信通过基础的私信方式传播的文件一定程度上保证了您与我们的联系性。
% 若您通过某些途径获得此文件,希望你能成为遵守规则的人,祝您前程似锦,工作顺利,生活愉快!
% Jade Moon 与你同在!



% ----------------------------------------------------- 
%	设定
% ----------------------------------------------------- 


% ----------------------------------------------------- 
% 相较于最初的模板,我去掉了边框,去掉下面注释就能加回去。
%%	加边框的命令
%%	参考:https://tex.stackexchange.com/questions/531559/how-to-add-the-page-border-for-first-two-pages-in-latex
%\usepackage{tikz}
%\usetikzlibrary{calc}
%\usepackage{eso-pic}
%\AddToShipoutPictureBG{%
	%	\begin{tikzpicture}[overlay,remember picture]
		%		\draw[line width=0.6pt] % 边框粗细
		%		($ (current page.north west) + (0.6cm,-0.6cm) $)
		%		rectangle
		%		($ (current page.south east) + (-0.6cm,0.6cm) $); % 边框位置
		%\end{tikzpicture}}


% ----------------------------------------------------- 
% 自定义颜色
\usepackage{xcolor}
\definecolor{fblack}{HTML}{3E324A} % #3e324a(紫黑)
\definecolor{fgrayblue}{HTML}{5A6B84} % 灰蓝
\definecolor{fgraygreen}{HTML}{1C2F62} % #1c2f62 (物天)
\definecolor{fred}{HTML}{A52A2A} % 红蓝cp
\definecolor{fsilver}{HTML}{E6E4E0} % #e6e4e0(银白)
\definecolor{fbluegreen}{HTML}{BCC7D6} % 用于高亮
\definecolor{flameorange}{HTML}{D94352} % 红蓝cp


% ----------------------------------------------------- 
% 链接
\usepackage{ctex} % 字体
\usepackage[top=28mm,bottom=28mm,left=25mm,right=25mm]{geometry} % 最通用的边距设置:上下边距:28mm;左右边距:25mm
\usepackage{hyperref} 
\hypersetup{
	colorlinks,
	linktoc = section, % 超链接位置,选项有section, page, all
	linkcolor = fgrayblue, % linkcolor 目录颜色
	citecolor = fgrayblue,  % citecolor 引用颜色
	urlcolor = fgrayblue % 外部链接
}


% ----------------------------------------------------- 
% 引入必要的包
%A
\usepackage{amsmath,enumerate,multirow,float} % 1.提供增强的数学公式排版功能,如 align、gather 等环境。2.增强 enumerate 环境,允许指定编号格式。3.在表格中合并多行单元格。4.用于控制浮动体(如表格、图片)的放置,提供 H 选项让浮动体固定在当前位置。
\usepackage{tabularx} % 创建自动调整列宽的表格,比 tabular 更灵活。
\usepackage{fancyhdr} % 自定义页眉和页脚。
\usepackage{graphicx} % 插图
\usepackage{appendix} % 创建附录部分,允许 \appendix 轻松管理附录编号。
\usepackage{geometry}  % 页面调整
\usepackage{colortbl} % 给表格填充颜色
%B
\usepackage{lipsum} % 生成占位文本 \lipsum[1-3]。
\usepackage{adjustbox}% 调整表格大小
\usepackage{tabularray} % 绘制表格时可以更加方便添加框线
\usepackage{authblk} % 管理多作者信息。
\usepackage{wrapfig} % 让图片环绕文本。
\usepackage{subfig} % 创建多个子图。
%C
\usepackage{amsfonts} % 提供 \mathbb{}、\mathcal{} 等数学字体。
\usepackage{mathrsfs} % 提供花体数学字体,如 \mathscr{}。
\usepackage{calligra} % 提供书法风格字体,可用于 \textcalligra{}。
\usepackage{caption} % 自定义图片/表格标题格式。
\usepackage{physics} % 提供数学物理相关命令,如 \bra{}、\ket{}。
\usepackage{braket} % 提供 \braket{a|b} 量子力学符号。
\usepackage{amssymb} % 包含更多数学符号

\usepackage{booktabs}


% ----------------------------------------------------- 
% 定义彩色环境
\usepackage{tcolorbox}
\tcbuselibrary{skins,breakable}

% 思考题
\newtcolorbox[auto counter,number within=section]{question}[1][]{
	breakable,
	enhanced, %跨页后不会显示下边框
	enhanced, breakable,
	%colback=fgrayblue!20!white,colframe=fgrayblue!80!black,colbacktitle = fgrayblue!80!black,
	colback=fgrayblue!20!white,colframe=fgrayblue,colbacktitle = fgrayblue, % 试试这个
	attach boxed title to top left = {yshift = -2mm, xshift = 5mm},
	boxed title style = {sharp corners},
	title={Reflection Question~\thetcbcounter:\quad},
	fonttitle=\bfseries,
	drop fuzzy shadow,
	#1
}

% 高亮
\newtcbox{\highlight}[1][fbluegreen]
{on line, arc = 0pt, outer arc = 0pt,
	colback = #1!85!white, colframe = white,
	boxsep = 0pt, left = 1pt, right = 1pt, top = 2pt, bottom = 2pt,
	boxrule = 0pt, bottomrule = 1pt, toprule = 1pt}

% 通用框架
\newtcolorbox{ubox}[2][]
{enhanced, breakable,
	colback = white, colframe = fbluegreen, coltitle = black,colbacktitle = fbluegreen,
	attach boxed title to top left = {yshift = -2mm, xshift = 5mm},
	boxed title style = {sharp corners},
	fonttitle = \bfseries,
	title={#2},#1}
	
% 代码块(仿MacOS的macbox)
\definecolor{wt1}{HTML}{ebebeb} % 白色主题的标题框顶部背景颜色
\definecolor{wt2}{HTML}{bebebe} % 白色主题的标题框底部背景颜色
\definecolor{wt3}{HTML}{efefef} % 白色主题的正文部分的背景颜色
\definecolor{circ1}{HTML}{eb605b} % 标题框左边第一个圆圈的颜色
\definecolor{circ2}{HTML}{f6bb31} % 标题框左边第二个圆圈的颜色
\definecolor{circ3}{HTML}{56cb45} % 标题框左边第三个圆圈的颜色
\definecolor{macosbox@bord}{RGB}{182,176,176}
\newtcolorbox{macbox}[2][]{
	enhanced,
	breakable,
	coltitle=black,
	colback = wt3,%macosbox@bg,
	boxrule=0mm,
	frame style={draw=macosbox@bord,fill=macosbox@bord},
	title style={top color=wt1,bottom color=wt2},
	drop fuzzy shadow=black,
	title={{\textcolor{circ1}{\huge$\bullet$}
			\textcolor{circ2}{\huge$\bullet$}
			\textcolor{circ3}{\huge$\bullet$}
			\hspace*{\fill}\texttt{#2}\hspace*{65mm}\hspace*{\fill}}},#1
}


% ---------------------------------------------------------------------
%	利用cleveref改变引用格式,\cref是引用命令
\usepackage{cleveref}
\crefformat{figure}{#2{\textcolor{fred}{\textbf{图 #1}}}#3} % 图片的引用格式
\crefformat{equation}{#2{(\textcolor{fred}{(#1)})}#3} % 公式的引用格式
\crefformat{table}{#2{\textcolor{fred}{\textbf{表 #1}}}#3} % 表格的引用格式
\crefformat{enumi}{#2{\textcolor{fred}{\textbf{[#1]}}}#3} % 文献的引用格式


% ---------------------------------------------------------------------
%	页眉页脚设置
\fancypagestyle{plain}{\pagestyle{fancy}}
\pagestyle{fancy}
\fancyhf{} % 清空默认页眉页脚
% 去掉页眉线(不想要页眉线把这行设为0pt,想要就改成这个\renewcommand{\headrule}{\color{fgraygreen}\hrule width\headwidth height 2pt})
\renewcommand{\headrulewidth}{0pt}
% 设置页脚线的颜色和粗细
\renewcommand{\footrule}{\color{fgraygreen}\hrule width\textwidth height 2pt}
% 自定义页眉
%\fancyhead[R]{\textcolor{fgrayblue}{电阻热噪声玻尔兹曼常量测量}}
% 自定义页脚内容
\fancyfoot[L]{\textcolor{fgrayblue}{黄罗琳,戴鹏辉,杨舒云,丁侯凯}}
\fancyfoot[R]{\textcolor{fgrayblue}{\textbf{\thepage}}}


% ---------------------------------------------------------------------
%	对目录、章节标题的设置
\renewcommand{\contentsname}{\centerline{\Huge 目录}}
\usepackage{titlesec}
\usepackage{titletoc}

% 【注意这里】由于学院logo实际上挺大的,有几百kb,所以你发现有点跑不动的话,你就把下面的section设置注释掉,换成这个
%\titleformat{\section}
%{\normalfont\bfseries\color{fgraygreen}\huge}
%{\thesection}
%{0.618}
%{}

% \titleformat{章节}[形状]{格式}{标题序号}{序号与标题间距}{标题前命令}[标题后命令]
\titleformat{\section}
{\normalfont\bfseries\color{fgraygreen}\huge}
{$\liuyin$}
{0.7em}
{}
\titleformat{\subsection}
{\normalfont\bfseries\color{fgrayblue}\LARGE} % 设置字体、大小、颜色
{\thesubsection} % 设置subsection编号的格式
{0.618em} % 编号和标题之间的间距
{}
\setcounter{secnumdepth}{3}
\titleformat{\subsubsection}
{\normalfont\bfseries\color{fgrayblue}\Large} 
{\thesubsubsection} 
{0.618em} 
{}


% ---------------------------------------------------------------------
% 图片、表格的设置(名称)
\captionsetup[figure]{labelfont={color=fgrayblue,bf},name=图} % 图片形式
\captionsetup[table]{labelfont={color=fgrayblue,bf},name=表} % 表格形式


% ---------------------------------------------------------------------
%   listing代码环境设置(不太好,将就用,有更好的可以自己改,其实一般也用不上)
\usepackage{listings}
\lstloadlanguages{python}
\lstdefinestyle{pythonstyle}{
	%backgroundcolor=\color{gray!6},% 用macbox就注释掉这一行
	language=python,
	frameround=tftt,
	%frame=shadowbox, % 用macbox就注释掉这一行
	keepspaces=true,
	breaklines,
	columns=spaceflexible,      
	basicstyle=\ttfamily\color{fblack}, % 基本文本设置(这里可以改改字号)
	keywordstyle=[1]\color{Yellow!75!black}\bfseries, 
	keywordstyle=[2]\color{flameorange!90!black},   
	stringstyle=\color{Purple!80!fblack},       
	showstringspaces=false,
	commentstyle=\ttfamily\color{Green!65!black},% 注释文本设置
	tabsize=2,
	morekeywords={as},
	morekeywords=[2]{np, plt, sp},
	numbers=left, % 代码行数
	numberstyle=\scriptsize\color{fgrayblue}, % 代码行数的数字字体设置
	stepnumber=1,
	numbersep=0pt % 代码行号和代码主体之间的距离,不用macbox就注释掉这一行
	%rulesepcolor=\color{fsilver} % 用macbox就注释掉这一行
}


% ---------------------------------------------------------------------
%	其他设置
\def\degree{${}^{\circ}$} % 角度
\graphicspath{{./images/}} % 插入图片的相对路径
\allowdisplaybreaks[4]  %允许公式跨页


% ---------------------------------------------------------------------
% 列表的设置
\usepackage{enumitem}
% 设置有序列表格式
\setlist[enumerate,1]{label=\textcolor{fblack}{\arabic*.}, font=\bfseries\color{fblack}}
\setlist[enumerate,2]{label=\textcolor{fblack}{(\arabic*)}, font=\bfseries\color{fblack}}
\setlist[enumerate,3]{label=\textcolor{fblack}{\roman*.}, font=\bfseries\color{fblack}}
\setlist[enumerate,4]{label=\textcolor{fblack}{(\roman*)}, font=\bfseries\color{fblack}}

%% 设置无序列表格式
\setlist[itemize,1]{label=\textcolor{fbluegreen}{$\blacktriangleright$}, font=\bfseries\color{fbluegreen}}
\setlist[itemize,2]{label=\textcolor{fbluegreen}{$\bullet$}, font=\bfseries\color{fbluegreen}}
\setlist[itemize,3]{label=\textcolor{fbluegreen}{$\blacksquare$}, font=\bfseries\color{fbluegreen}}

% 【注意这里】由于学院logo实际上挺大的,有几百kb,所以你发现有点跑不动的话,你就把下面的无序列表设置注释掉,换成上面那个
%\setlist[itemize,1]{label=$\liuyin$}
%\setlist[itemize,2]{label=\textcolor{fred}{$\blacktriangleright$}, font=\bfseries\color{fred}}
%\setlist[itemize,3]{label=\textcolor{fbluegreen}{$\bullet$}, font=\bfseries\color{fbluegreen}}


% ---------------------------------------------------------------------
% 封装
% 实验信息
\usepackage{ifthen}  
\newcommand{\infoTable}[9]{
    \begin{flushleft}
        \Huge \textcolor{fgraygreen}{\textbf{\kaishu #1}}  % 使用 ##1 来引用参数
    \end{flushleft}

    \vspace{-0.3cm}

    \begin{table}[h!]
        \textnormal{\textcolor{fgraygreen}{\rule{0.75\textwidth}{1.53pt} }}\\
        \begin{tabularx}{0.7\textwidth}{p{0.175\textwidth}p{0.525\textwidth}}
            \large\textcolor{fgraygreen}{\textbf{实验时间:}} &  \textbf{#2}\\
            \textcolor{fgraygreen}{\textbf{实验地点:}} &  \textbf{#3}\\
            \textcolor{fgraygreen}{\textbf{环境信息:}} &  \textbf{#4} \\
            \textcolor{fgraygreen}{\textbf{实验人1:}} &  \textbf{#5} \\
            \textcolor{fgraygreen}{\textbf{实验人2:}} &  \textbf{#6} \\
            \textcolor{fgraygreen}{\textbf{实验人3:}} &  \textbf{#7} \\
            \textcolor{fgraygreen}{\textbf{实验人4:}} &  \textbf{#8} \\
            \textcolor{fgraygreen}{\textbf{指导老师}} &  \textbf{#9}\\
        \end{tabularx}\\
        \textnormal{\textcolor{fgraygreen}{\rule{0.75\textwidth}{1.5pt} }}
    \end{table}
}

% 定义实验信息
\newcommand{\experimentNumber}[1]{\def\currentExperiment{#1}}  % 设置实验序号

% 定义页眉
\newcommand{\setExperimentHeader}{
    \ifthenelse{\equal{\currentExperiment}{E1}}{
        \fancyhead[R]{低温技术平台与高温超导研究} % 设置页眉为实验E1的名称
    }{
    \ifthenelse{\equal{\currentExperiment}{E2}}{
        \fancyhead[R]{ECDL外腔式半导体激光器实验} % 设置页眉为实验E2的名称
    }{
    \ifthenelse{\equal{\currentExperiment}{E3}}{
        \fancyhead[R]{光泵磁共振实验} % 设置页眉为实验E3的名称
    }{
    \ifthenelse{\equal{\currentExperiment}{E5}}{
        \fancyhead[R]{双光子纠缠源研究和量子非局域性验证实验} % 设置页眉为实验E5的名称
    }{
    \ifthenelse{\equal{\currentExperiment}{E6}}{
        \fancyhead[R]{散射光成像实验} % 设置页眉为实验E6的名称
    }{
        % 如果实验序号无效,则设置默认页眉
        \fancyhead[R]{实验未找到}
    }
    }}}}
}

% 定义生成实验信息表格的命令
\newcommand{\generateExperimentInfo}{
    \ifthenelse{\equal{\currentExperiment}{E1}}{
        % E1 实验信息
        \infoTable{E1: 低温技术平台与高温超导研究}
        {2025年3月21日}
        {教学楼物理实验室A101}
        {室温26\degree}
        {黄罗琳 22344001}
        {戴鹏辉 22344016} 
        {杨舒云 22344020} 
        {丁侯凯 22344009} 
        {指导老师}
    }{
    \ifthenelse{\equal{\currentExperiment}{E3}}{
        % E3 实验信息
        \infoTable{E3: 光泵磁共振实验}
        {2025年5月16日}
        {A407光学实验室}
        {室温25\degree}
        {黄罗琳 22344001}
        {戴鹏辉 22344016} 
        {杨舒云 22344020} 
        {丁侯凯 22344009} 
        {指导老师}
    }{
    \ifthenelse{\equal{\currentExperiment}{E5}}{
        % E5 实验信息
        \infoTable{E5: 双光子纠缠源研究和量子非局域性验证实验}
        {2025年3月07日}
        {A102暗室}
        {室温24\degree}
        {黄罗琳 22344001}
        {戴鹏辉 22344016} 
        {杨舒云 22344020} 
        {丁侯凯 22344009} 
        {指导老师}
    }{
    \ifthenelse{\equal{\currentExperiment}{E6}}{
        % E6 实验信息
        \infoTable{E6: 散射光成像实验}
        {2025年5月30日}
        {A102}
        {室温23\degree}
        {黄罗琳 22344001}
        {戴鹏辉 22344016} 
        {杨舒云 22344020} 
        {丁侯凯 223440XX} 
        {指导老师}
    }
    }}}
}



% 定制的节
\newcommand{\JMSection}[1]{
	\section{#1}
	\vspace{-0.7cm} % 负值缩小间距
	\noindent\textcolor{fgraygreen}{\rule{0.382\textwidth}{2pt} }
	\vspace{7pt}
}

% 个人信息表(一人一表)
\newcommand{\infoPersonal}[6]{
	\renewcommand\arraystretch{1.4}
	\begin{tabularx}{\textwidth}{|X|X|X|X|}
		\hline
		专业: & #1 &年级: & #2\\
		\hline
		学生姓名: & #3 & 学号: & #4  \\
		\hline
		实验: & #5 & 日期: & #6\\
		\hline
	\end{tabularx}
}

% 粗糙的、用于偷懒的参考文献条目(中文版)
\newcommand{\JMRef}[7]{
	\item\label{ref:#1} #2\quad#3[#4]. \emph{#5}, #6, #7.
}

% 添加了学院配套主题图标
\newcommand{\liuyin}{\mathord{\raisebox{-0.2ex}{\includegraphics[height=0.8em]{images/theme/sysuspa_icon.png}}}}

% 强调
\newcommand{\JMEmph}[1]{
	\textbf{\textcolor{flameorange}{#1}}
}

% 加了一个评分表(来自原版)
\newcommand{\scoresTable}[8]{
	\begin{table}[h!]
		\renewcommand\arraystretch{1.7}
		\begin{tabularx}{\textwidth}{
				|X|X|X|X
				|X|X|X|X|}
			\hline
			\multicolumn{2}{|c|}{预习报告} & \multicolumn{2}{|c|}{实验记录} & \multicolumn{2}{|c|}{分析讨论} & \multicolumn{2}{|c|}{总成绩} \\
			\hline
			\centering#1&\centering#2 &\centering#3 &\centering#4 &\centering#5 &\centering#6 &\centering#7 &{\centering#8} \\
			\hline
		\end{tabularx}
	\end{table}
}


% ----------------------------------------------------- 
%	下面不用看,致敬原本的流萤主题
% ----------------------------------------------------- 


% ----------------------------------------------------- 
% 我还是不太能理解,为什么他们说今晚月色很美是含蓄的表白。 
% 直到我看到朝阳下江水漾起的片片金鳞、漆黑的夜空中不甘散去的橘黄云彩、亦或者是夜宵摊子上高谈阔论掺杂着串子被炭火炙烤出油香的烟火气息,都下意识拿出手机想跟你分享。 
% 我想把自己觉得美丽的东西传递给你。 
% 今天依然天晴,我亲爱的流萤。