% !TEX root = ../main.tex

% 前言
\clearpage
\thispagestyle{empty}


%---------------------------------------------------------------------
% 信息栏
% 格式:实验标题、时间、地点、环境、实验人、补充实验人(特定格式如下)和指导老师
% \textcolor{fgraygreen}{\textbf{实验人2:}} &  \textbf{姓名 学号} \\
\generateExperimentInfo


%---------------------------------------------------------------------
% 注意事项
\noindent\textcolor{fgraygreen}{\textbf{注意事项}}

实验报告由\textbf{预习}、\textbf{实验}和\textbf{分析与讨论}三部分组成,并附封面页与附件。预习报告要求课前深入研读实验手册,掌握实验原理,熟悉仪器设备及其使用方法,完成实验思考题,并提前准备实验记录表(可参考实验报告模板打印)。实验记录需客观、详细地记录实验条件、现象及数据,须使用圆珠笔或钢笔书写并签名(\textbf{铅笔记录无效})。\textbf{原始记录必须完整保留,包括错误和修改内容,错误更正需按标准程序进行}。实验记录不得录入计算机打印,但可扫描手写笔记后打印,实验结束前须经指导教师检查并签字。数据处理与分析环节需对原始数据进行处理(除以仪器学习为主的实验外),评估数据的可靠性和合理性,并以标准格式呈现(图表需编号并规范引用)。此外,还需分析实验现象,回答实验思考题,规范引用数据,并最终得出实验结论。实验报告需在实验结束后一周内提交(特殊情况不超过两周)。

	
%---------------------------------------------------------------------	
\noindent\textcolor{fgraygreen}{\rule{\textwidth}{1.5pt} }

% 特别说明
\noindent\textcolor{fgraygreen}{\textbf{特别说明}}

\textbf{本实验报告模板基于 MIT License 许可协议进行分发和使用,使用本模板即表示您同意遵守相关条款。}
本模板由组内成员\textbf{pifuyuini}与\textbf{Jade Moon}共同开发,随着我们步入高年级阶段的近代物理实验课程,实验报告的书写要求也随之提升——不再局限于套用固定格式的模板,而是更加注重参考学术论文的结构与风格来完成实验报告,这也是本模板设计的初衷。因此,与基础物理实验课程中学院提供的“标准模板”相比,本报告在架构和行文风格上均作出了一定的调整。\textbf{如本报告的格式或内容与老师或助教的阅读习惯存在差异,敬请谅解。}

%---------------------------------------------------------------------
\noindent\textcolor{fgraygreen}{\rule{\textwidth}{1.5pt} }

% 评分栏

% 另一种风格的评分表
%\begin{table}[h!]
%	\renewcommand{\arraystretch}{1.2} % 行距
%	\begin{tabular}{|m{8cm}|m{7cm}|}
%		\hline
%		评分项目 & 得分 \\
%		\hline
%		预习(Preparation) & \\
%		\hline
%		实验(Experiment) & \\
%		\hline
%		报告(Report) & \\
%		\hline
%		总分(Total Score) & \\
%		\hline
%		教师签名(Signature) & \\
%		\hline
%	\end{tabular}
%\end{table}
	

	

%---------------------------------------------------------------------	
% 目录
\clearpage
\tableofcontents	
