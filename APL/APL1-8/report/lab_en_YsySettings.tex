% 设定说明


% ----------------------------------------------------- 
% 加边框的命令
%参考:https://tex.stackexchange.com/questions/531559/how-to-add-the-page-border-for-first-two-pages-in-latex
\usepackage{tikz}
\usetikzlibrary{calc}
\usepackage{eso-pic}
\AddToShipoutPictureBG{%
\begin{tikzpicture}[overlay,remember picture]
\draw[line width=0.6pt] % 边框粗细
    ($ (current page.north west) + (0.6cm,-0.6cm) $)
    rectangle
    ($ (current page.south east) + (-0.6cm,0.6cm) $); % 边框位置
\end{tikzpicture}}


% ----------------------------------------------------- 
% 自定义颜色(这里采用的是我老婆的配色!)
\usepackage{xcolor}
\definecolor{fblack}{HTML}{3E324A} % #3e324a(紫黑)
\definecolor{fgrayblue}{HTML}{475D7B} % #475d7b(灰蓝)
\definecolor{fgraygreen}{HTML}{97C6C0} % #97c6c0(灰绿)
\definecolor{fred}{HTML}{E26E1B} % #e26e1b(深橘红)
\definecolor{fsilver}{HTML}{E6E4E0} % #e6e4e0(银白)
\definecolor{fbluegreen}{HTML}{4DF8E8} % #4df8e8(蓝绿)


% ----------------------------------------------------- 
% 链接
\usepackage{ctex}
\usepackage[top=28mm,bottom=28mm,left=15mm,right=15mm]{geometry}
\usepackage{hyperref} 
\hypersetup{
	colorlinks,
	linktoc = section, % 超链接位置,选项有section, page, all
	linkcolor = fgrayblue, % linkcolor 目录颜色
	citecolor = fgrayblue  % citecolor 引用颜色
}


% ----------------------------------------------------- 
% 引入必要的包
\usepackage{amsmath,enumerate,multirow,float}
\usepackage{tabularx}
\usepackage{tabu}
\usepackage{subfig}
\usepackage{fancyhdr}
\usepackage{graphicx}
\usepackage{wrapfig}  
\usepackage{physics}
\usepackage{appendix}
\usepackage{amsfonts}
%B
\usepackage{mathrsfs} % 字体
\usepackage{calligra}
\usepackage{lipsum}
\usepackage{adjustbox}% 调整表格大小
\usepackage{tabularray} % 绘制表格时可以更加方便添加框线
%C
\usepackage{caption}
\usepackage{authblk}
\usepackage{braket}
\usepackage{amssymb} % 包含更多数学符号


% ----------------------------------------------------- 
% 定义彩色环境
\usepackage{tcolorbox}
\tcbuselibrary{skins,breakable}

% 思考题
\newtcolorbox[auto counter,number within=section]{question}[1][]{
	breakable,
	enhanced, %跨页后不会显示下边框
	enhanced, breakable,
	colback=fbluegreen!20!white,colframe=fbluegreen!80!black,colbacktitle = fbluegreen!80!black,
	attach boxed title to top left = {yshift = -2mm, xshift = 5mm},
	boxed title style = {sharp corners},
	title={Reflection Question~\thetcbcounter:\quad},
	fonttitle=\bfseries,
	drop fuzzy shadow,
	#1
}
%% 这是原版的,现在已经被注释掉了,不用看---------------------------------
%\newtcolorbox[auto counter,number within=section]{question}[1][]{
%  top=2pt,bottom=2pt,arc=1mm,
%  boxrule=0.5pt,
%%   frame hidden,
%  breakable,
%  enhanced, %跨页后不会显示下边框
%  coltitle=fgrayblue!80!gray,
%  colframe=fbluegreen,
%  colback=fgraygreen!3!white,
%  drop fuzzy shadow,
%  title={Reflection Question~\thetcbcounter:\quad},
%  fonttitle=\bfseries,
%  attach title to upper,
%  #1
%}
%% 这是原版的,现在已经被注释掉了,不用看---------------------------------

% 高亮
\newtcbox{\highlight}[1][fbluegreen]
{on line, arc = 0pt, outer arc = 0pt,
	colback = #1!85!white, colframe = white,
	boxsep = 0pt, left = 1pt, right = 1pt, top = 2pt, bottom = 2pt,
	boxrule = 0pt, bottomrule = 1pt, toprule = 1pt}

% 通用框架
\newtcolorbox{ubox}[2][]
{enhanced, breakable,
	colback = white, colframe = fbluegreen, coltitle = black,colbacktitle = fbluegreen,
	attach boxed title to top left = {yshift = -2mm, xshift = 5mm},
	boxed title style = {sharp corners},
	fonttitle = \bfseries,
	title={#2},#1}

% ---------------------------------------------------------------------
%	利用cleveref改变引用格式,\cref是引用命令
\usepackage{cleveref}
\crefformat{figure}{#2{\textcolor{fred}{\textbf{Figure #1}}}#3} % 图片的引用格式
\crefformat{equation}{#2{(\textcolor{fred}{(#1)})}#3} % 公式的引用格式
\crefformat{table}{#2{\textcolor{fred}{\textbf{Table #1}}}#3} % 表格的引用格式


% ---------------------------------------------------------------------
%	页眉页脚设置
\fancypagestyle{plain}{\pagestyle{fancy}}
\pagestyle{fancy}
\lhead{SYSU.SPA \quad \textbf{Advanced Physics Laboratory \uppercase\expandafter{\romannumeral1}}} % 左边页眉
\rhead{杨舒云、戴鹏辉、朱政鑫、马万成} % 右边页眉
\cfoot{\thepage} % 页脚,中间添加页码


% ---------------------------------------------------------------------
%	对目录、章节标题的设置
\renewcommand{\contentsname}{\centerline{\Huge TABLE \quad OF \quad CONTENTS}}
\usepackage{titlesec}
\usepackage{titletoc}
% \titleformat{章节}[形状]{格式}{标题序号}{序号与标题间距}{标题前命令}[标题后命令]
\titleformat{\section}{\normalfont\bfseries\centering\color{fgrayblue}\LARGE}{}{1em}{}
\titleformat{\subsection}
{\normalfont\bfseries\color{fgrayblue}\Large} % 设置字体、大小、颜色
{\thesubsection} % 设置subsection编号的格式
{1em} % 编号和标题之间的间距
{}
\setcounter{secnumdepth}{3}
\titleformat{\subsubsection}
{\normalfont\bfseries\color{fgrayblue}\large} % 设置字体、大小、颜色
{\thesubsubsection} % 设置subsubsection编号的格式
{1em} % 编号和标题之间的间距
{}

% ---------------------------------------------------------------------
% 图片、表格的设置
\captionsetup[figure]{labelfont={color=fgrayblue,bf},name=Figure} % 图片形式
\captionsetup[table]{labelfont={color=fgrayblue,bf},name=Table} % 表格形式


% ---------------------------------------------------------------------
%   listing代码环境设置
\usepackage{listings}
\lstloadlanguages{python}
\lstdefinestyle{pythonstyle}{
	backgroundcolor=\color{gray!6},
	language=python,
	frameround=tftt,
	frame=shadowbox, 
	keepspaces=true,
	breaklines,
	columns=spaceflexible,      
	basicstyle=\ttfamily\color{fblack}, % 基本文本设置(这里可以改改字号)
	keywordstyle=[1]\color{fbluegreen!90!black}\bfseries, 
	keywordstyle=[2]\color{fred!90!black},   
	stringstyle=\color{Purple!80!fblack},       
	showstringspaces=false,
	commentstyle=\ttfamily\color{fgraygreen!65!black},% 注释文本设置
	tabsize=2,
	morekeywords={as},
	morekeywords=[2]{np, plt, sp},
	numbers=left, % 代码行数
	numberstyle=\scriptsize\color{fgrayblue}, % 代码行数的数字字体设置
	stepnumber=1,
	rulesepcolor=\color{fsilver}
}


% ---------------------------------------------------------------------
%	其他设置
\def\degree{${}^{\circ}$} % 角度
\graphicspath{{./images/}} % 插入图片的相对路径
\allowdisplaybreaks[4]  %允许公式跨页


% ---------------------------------------------------------------------
% 列表的设置
\usepackage{enumitem}
% 设置有序列表格式
\setlist[enumerate,1]{label=\textcolor{fblack}{\arabic*.}, font=\bfseries\color{fblack}}
\setlist[enumerate,2]{label=\textcolor{fblack}{(\arabic*)}, font=\bfseries\color{fblack}}
\setlist[enumerate,3]{label=\textcolor{fblack}{\roman*.}, font=\bfseries\color{fblack}}
\setlist[enumerate,4]{label=\textcolor{fblack}{(\roman*)}, font=\bfseries\color{fblack}}

% 设置无序列表格式
\setlist[itemize,1]{label=\textcolor{fbluegreen}{$\blacktriangleright$}, font=\bfseries\color{fbluegreen}}
\setlist[itemize,2]{label=\textcolor{fbluegreen}{$\bullet$}, font=\bfseries\color{fbluegreen}}
\setlist[itemize,3]{label=\textcolor{fbluegreen}{$\blacksquare$}, font=\bfseries\color{fbluegreen}}