% 相关版权与传播说明

% 本文档基于 MIT License 进行许可,部分内容参考来源:https://huanyushi.github.io/posts/labreport-template/ 和 https://www.yykspace.com/cn/index.html,非完全原创。

%!TEX program = xelatex
\documentclass[dvipsnames, svgnames, a4paper, 11pt]{article} 
% 指定文档类型为 article,使用 A4 纸张,字体大小为 11pt,并启用 dvipsnames 和 svgnames 颜色名称,以支持多种颜色的使用。

%---------------------------------------------------------------------
% 设定
%---------------------------------------------------------------------
\let\icon\liuyin  
\let\JMSection\ysySection
\let\JMRef\ysyRef
\let\JMEmph\ysyEmph
% ----------------------------------------------------- 
%	说明
% ----------------------------------------------------- 

% 本模板基于YSY的优秀实验报告模板进行优化,其完成了绝大部分工作,在此表示衷心感谢其支持与贡献。
% 无论您通过何种途径获得本文件,务必确保文件内容的隐私得到充分保护。请注意,本文件不支持上传任何平台,若因文件内容引发任何权益纠纷,文件的基础参与建设者将不承担任何责任。
% 若您愿意与我们进行进一步交流,欢迎通过以下网址与我们联系:https:// 。我们将在与相关人员沟通后,为您提供进一步的技术支持,我们相信通过基础的私信方式传播的文件一定程度上保证了您与我们的联系性。
% 若您通过某些途径获得此文件,希望你能成为遵守规则的人,祝您前程似锦,工作顺利,生活愉快!
% Jade Moon 与你同在!



% ----------------------------------------------------- 
%	设定
% ----------------------------------------------------- 


% ----------------------------------------------------- 
% 相较于最初的模板,我去掉了边框,去掉下面注释就能加回去。
%%	加边框的命令
%%	参考:https://tex.stackexchange.com/questions/531559/how-to-add-the-page-border-for-first-two-pages-in-latex
%\usepackage{tikz}
%\usetikzlibrary{calc}
%\usepackage{eso-pic}
%\AddToShipoutPictureBG{%
	%	\begin{tikzpicture}[overlay,remember picture]
		%		\draw[line width=0.6pt] % 边框粗细
		%		($ (current page.north west) + (0.6cm,-0.6cm) $)
		%		rectangle
		%		($ (current page.south east) + (-0.6cm,0.6cm) $); % 边框位置
		%\end{tikzpicture}}


% ----------------------------------------------------- 
% 自定义颜色
\usepackage{xcolor}
\definecolor{fblack}{HTML}{3E324A} % #3e324a(紫黑)
\definecolor{fgrayblue}{HTML}{5A6B84} % 灰蓝
\definecolor{fgraygreen}{HTML}{1C2F62} % #1c2f62 (物天)
\definecolor{fred}{HTML}{A52A2A} % 红蓝cp
\definecolor{fsilver}{HTML}{E6E4E0} % #e6e4e0(银白)
\definecolor{fbluegreen}{HTML}{BCC7D6} % 用于高亮
\definecolor{flameorange}{HTML}{D94352} % 红蓝cp


% ----------------------------------------------------- 
% 链接
\usepackage{ctex} % 字体
\usepackage[top=28mm,bottom=28mm,left=25mm,right=25mm]{geometry} % 最通用的边距设置:上下边距:28mm;左右边距:25mm
\usepackage{hyperref} 
\hypersetup{
	colorlinks,
	linktoc = section, % 超链接位置,选项有section, page, all
	linkcolor = fgrayblue, % linkcolor 目录颜色
	citecolor = fgrayblue,  % citecolor 引用颜色
	urlcolor = fgrayblue % 外部链接
}


% ----------------------------------------------------- 
% 引入必要的包
%A
\usepackage{amsmath,enumerate,multirow,float} % 1.提供增强的数学公式排版功能,如 align、gather 等环境。2.增强 enumerate 环境,允许指定编号格式。3.在表格中合并多行单元格。4.用于控制浮动体(如表格、图片)的放置,提供 H 选项让浮动体固定在当前位置。
\usepackage{tabularx} % 创建自动调整列宽的表格,比 tabular 更灵活。
\usepackage{fancyhdr} % 自定义页眉和页脚。
\usepackage{graphicx} % 插图
\usepackage{appendix} % 创建附录部分,允许 \appendix 轻松管理附录编号。
\usepackage{geometry}  % 页面调整
\usepackage{colortbl} % 给表格填充颜色
%B
\usepackage{lipsum} % 生成占位文本 \lipsum[1-3]。
\usepackage{adjustbox}% 调整表格大小
\usepackage{tabularray} % 绘制表格时可以更加方便添加框线
\usepackage{authblk} % 管理多作者信息。
\usepackage{wrapfig} % 让图片环绕文本。
\usepackage{subfig} % 创建多个子图。
%C
\usepackage{amsfonts} % 提供 \mathbb{}、\mathcal{} 等数学字体。
\usepackage{mathrsfs} % 提供花体数学字体,如 \mathscr{}。
\usepackage{calligra} % 提供书法风格字体,可用于 \textcalligra{}。
\usepackage{caption} % 自定义图片/表格标题格式。
\usepackage{physics} % 提供数学物理相关命令,如 \bra{}、\ket{}。
\usepackage{braket} % 提供 \braket{a|b} 量子力学符号。
\usepackage{amssymb} % 包含更多数学符号
\usepackage{amsmath} % 用于数学符号,特别是下标
\usepackage{booktabs} % 用于表格的好看线条


% ----------------------------------------------------- 
% 定义彩色环境
\usepackage{tcolorbox}
\tcbuselibrary{skins,breakable}

% 思考题
\newtcolorbox[auto counter,number within=section]{question}[1][]{
	breakable,
	enhanced, %跨页后不会显示下边框
	enhanced, breakable,
	%colback=fgrayblue!20!white,colframe=fgrayblue!80!black,colbacktitle = fgrayblue!80!black,
	colback=fgrayblue!20!white,colframe=fgrayblue,colbacktitle = fgrayblue, % 试试这个
	attach boxed title to top left = {yshift = -2mm, xshift = 5mm},
	boxed title style = {sharp corners},
	title={Reflection Question~\thetcbcounter:\quad},
	fonttitle=\bfseries,
	drop fuzzy shadow,
	#1
}

% 高亮
\newtcbox{\highlight}[1][fbluegreen]
{on line, arc = 0pt, outer arc = 0pt,
	colback = #1!85!white, colframe = white,
	boxsep = 0pt, left = 1pt, right = 1pt, top = 2pt, bottom = 2pt,
	boxrule = 0pt, bottomrule = 1pt, toprule = 1pt}

% 通用框架
\newtcolorbox{ubox}[2][]
{enhanced, breakable,
	colback = white, colframe = fbluegreen, coltitle = black,colbacktitle = fbluegreen,
	attach boxed title to top left = {yshift = -2mm, xshift = 5mm},
	boxed title style = {sharp corners},
	fonttitle = \bfseries,
	title={#2},#1}
	
% 代码块(仿MacOS的macbox)
\definecolor{wt1}{HTML}{ebebeb} % 白色主题的标题框顶部背景颜色
\definecolor{wt2}{HTML}{bebebe} % 白色主题的标题框底部背景颜色
\definecolor{wt3}{HTML}{efefef} % 白色主题的正文部分的背景颜色
\definecolor{circ1}{HTML}{eb605b} % 标题框左边第一个圆圈的颜色
\definecolor{circ2}{HTML}{f6bb31} % 标题框左边第二个圆圈的颜色
\definecolor{circ3}{HTML}{56cb45} % 标题框左边第三个圆圈的颜色
\definecolor{macosbox@bord}{RGB}{182,176,176}
\newtcolorbox{macbox}[2][]{
	enhanced,
	breakable,
	coltitle=black,
	colback = wt3,%macosbox@bg,
	boxrule=0mm,
	frame style={draw=macosbox@bord,fill=macosbox@bord},
	title style={top color=wt1,bottom color=wt2},
	drop fuzzy shadow=black,
	title={{\textcolor{circ1}{\huge$\bullet$}
			\textcolor{circ2}{\huge$\bullet$}
			\textcolor{circ3}{\huge$\bullet$}
			\hspace*{\fill}\texttt{#2}\hspace*{65mm}\hspace*{\fill}}},#1
}


% ---------------------------------------------------------------------
%	利用cleveref改变引用格式,\cref是引用命令
\usepackage{cleveref}
\crefformat{figure}{#2{\textcolor{fred}{\textbf{图 #1}}}#3} % 图片的引用格式
\crefformat{equation}{#2{(\textcolor{fred}{(#1)})}#3} % 公式的引用格式
\crefformat{table}{#2{\textcolor{fred}{\textbf{表 #1}}}#3} % 表格的引用格式
\crefformat{enumi}{#2{\textcolor{fred}{\textbf{[#1]}}}#3} % 文献的引用格式


% ---------------------------------------------------------------------
%	页眉页脚设置
\fancypagestyle{plain}{\pagestyle{fancy}}
\pagestyle{fancy}
\fancyhf{} % 清空默认页眉页脚
% 去掉页眉线(不想要页眉线把这行设为0pt,想要就改成这个\renewcommand{\headrule}{\color{fgraygreen}\hrule width\headwidth height 2pt})
\renewcommand{\headrulewidth}{0pt}
% 设置页脚线的颜色和粗细
\renewcommand{\footrule}{\color{fgraygreen}\hrule width\textwidth height 2pt}
% 自定义页眉
%\fancyhead[R]{\textcolor{fgrayblue}{电阻热噪声玻尔兹曼常量测量}}
% 自定义页脚内容
\fancyfoot[L]{\textcolor{fgrayblue}{黄罗琳,戴鹏辉,杨舒云,丁侯凯}}
\fancyfoot[R]{\textcolor{fgrayblue}{\textbf{\thepage}}}


% ---------------------------------------------------------------------
%	对目录、章节标题的设置
\renewcommand{\contentsname}{\centerline{\Huge 目录}}
\usepackage{titlesec}
\usepackage{titletoc}

% 【注意这里】由于学院logo实际上挺大的,有几百kb,所以你发现有点跑不动的话,你就把下面的section设置注释掉,换成这个
%\titleformat{\section}
%{\normalfont\bfseries\color{fgraygreen}\huge}
%{\thesection}
%{0.618}
%{}

% \titleformat{章节}[形状]{格式}{标题序号}{序号与标题间距}{标题前命令}[标题后命令]
\titleformat{\section}
{\normalfont\bfseries\color{fgraygreen}\huge}
{$\liuyin$}
{0.7em}
{}
\titleformat{\subsection}
{\normalfont\bfseries\color{fgrayblue}\LARGE} % 设置字体、大小、颜色
{\thesubsection} % 设置subsection编号的格式
{0.618em} % 编号和标题之间的间距
{}
\setcounter{secnumdepth}{3}
\titleformat{\subsubsection}
{\normalfont\bfseries\color{fgrayblue}\Large} 
{\thesubsubsection} 
{0.618em} 
{}


% ---------------------------------------------------------------------
% 图片、表格的设置(名称)
\captionsetup[figure]{labelfont={color=fgrayblue,bf},name=图} % 图片形式
\captionsetup[table]{labelfont={color=fgrayblue,bf},name=表} % 表格形式


% ---------------------------------------------------------------------
%   listing代码环境设置(不太好,将就用,有更好的可以自己改,其实一般也用不上)
\usepackage{listings}
\lstloadlanguages{python}
\lstdefinestyle{pythonstyle}{
	%backgroundcolor=\color{gray!6},% 用macbox就注释掉这一行
	language=python,
	frameround=tftt,
	%frame=shadowbox, % 用macbox就注释掉这一行
	keepspaces=true,
	breaklines,
	columns=spaceflexible,      
	basicstyle=\ttfamily\color{fblack}, % 基本文本设置(这里可以改改字号)
	keywordstyle=[1]\color{Yellow!75!black}\bfseries, 
	keywordstyle=[2]\color{flameorange!90!black},   
	stringstyle=\color{Purple!80!fblack},       
	showstringspaces=false,
	commentstyle=\ttfamily\color{Green!65!black},% 注释文本设置
	tabsize=2,
	morekeywords={as},
	morekeywords=[2]{np, plt, sp},
	numbers=left, % 代码行数
	numberstyle=\scriptsize\color{fgrayblue}, % 代码行数的数字字体设置
	stepnumber=1,
	numbersep=0pt % 代码行号和代码主体之间的距离,不用macbox就注释掉这一行
	%rulesepcolor=\color{fsilver} % 用macbox就注释掉这一行
}


% ---------------------------------------------------------------------
%	其他设置
\def\degree{${}^{\circ}$} % 角度
\graphicspath{{./images/}} % 插入图片的相对路径
\allowdisplaybreaks[4]  %允许公式跨页


% ---------------------------------------------------------------------
% 列表的设置
\usepackage{enumitem}
% 设置有序列表格式
\setlist[enumerate,1]{label=\textcolor{fblack}{\arabic*.}, font=\bfseries\color{fblack}}
\setlist[enumerate,2]{label=\textcolor{fblack}{(\arabic*)}, font=\bfseries\color{fblack}}
\setlist[enumerate,3]{label=\textcolor{fblack}{\roman*.}, font=\bfseries\color{fblack}}
\setlist[enumerate,4]{label=\textcolor{fblack}{(\roman*)}, font=\bfseries\color{fblack}}

%% 设置无序列表格式
\setlist[itemize,1]{label=\textcolor{fbluegreen}{$\blacktriangleright$}, font=\bfseries\color{fbluegreen}}
\setlist[itemize,2]{label=\textcolor{fbluegreen}{$\bullet$}, font=\bfseries\color{fbluegreen}}
\setlist[itemize,3]{label=\textcolor{fbluegreen}{$\blacksquare$}, font=\bfseries\color{fbluegreen}}

% 【注意这里】由于学院logo实际上挺大的,有几百kb,所以你发现有点跑不动的话,你就把下面的无序列表设置注释掉,换成上面那个
%\setlist[itemize,1]{label=$\liuyin$}
%\setlist[itemize,2]{label=\textcolor{fred}{$\blacktriangleright$}, font=\bfseries\color{fred}}
%\setlist[itemize,3]{label=\textcolor{fbluegreen}{$\bullet$}, font=\bfseries\color{fbluegreen}}


% ---------------------------------------------------------------------
% 封装
% 实验信息
\usepackage{ifthen}  
\newcommand{\infoTable}[9]{
    \begin{flushleft}
        \Huge \textcolor{fgraygreen}{\textbf{\kaishu #1}}  % 使用 ##1 来引用参数
    \end{flushleft}

    \vspace{-0.3cm}

    \begin{table}[h!]
        \textnormal{\textcolor{fgraygreen}{\rule{0.75\textwidth}{1.53pt} }}\\
        \begin{tabularx}{0.7\textwidth}{p{0.175\textwidth}p{0.525\textwidth}}
            \large\textcolor{fgraygreen}{\textbf{实验时间:}} &  \textbf{#2}\\
            \textcolor{fgraygreen}{\textbf{实验地点:}} &  \textbf{#3}\\
            \textcolor{fgraygreen}{\textbf{环境信息:}} &  \textbf{#4} \\
            \textcolor{fgraygreen}{\textbf{实验人1:}} &  \textbf{#5} \\
            \textcolor{fgraygreen}{\textbf{实验人2:}} &  \textbf{#6} \\
            \textcolor{fgraygreen}{\textbf{实验人3:}} &  \textbf{#7} \\
            \textcolor{fgraygreen}{\textbf{实验人4:}} &  \textbf{#8} \\
            \textcolor{fgraygreen}{\textbf{指导老师}} &  \textbf{#9}\\
        \end{tabularx}\\
        \textnormal{\textcolor{fgraygreen}{\rule{0.75\textwidth}{1.5pt} }}
    \end{table}
}

% 定义实验信息
\newcommand{\experimentNumber}[1]{\def\currentExperiment{#1}}  % 设置实验序号

% 定义页眉
\newcommand{\setExperimentHeader}{
    \ifthenelse{\equal{\currentExperiment}{E1}}{
        \fancyhead[R]{低温技术平台与高温超导研究} % 设置页眉为实验E1的名称
    }{
    \ifthenelse{\equal{\currentExperiment}{E2}}{
        \fancyhead[R]{ECDL外腔式半导体激光器实验} % 设置页眉为实验E2的名称
    }{
    \ifthenelse{\equal{\currentExperiment}{E3}}{
        \fancyhead[R]{光泵磁共振实验} % 设置页眉为实验E3的名称
    }{
    \ifthenelse{\equal{\currentExperiment}{E5}}{
        \fancyhead[R]{双光子纠缠源研究和量子非局域性验证实验} % 设置页眉为实验E5的名称
    }{
    \ifthenelse{\equal{\currentExperiment}{E6}}{
        \fancyhead[R]{散射光成像实验} % 设置页眉为实验E6的名称
    }{
        % 如果实验序号无效,则设置默认页眉
        \fancyhead[R]{实验未找到}
    }
    }}}}
}

% 定义生成实验信息表格的命令
\newcommand{\generateExperimentInfo}{
    \ifthenelse{\equal{\currentExperiment}{E1}}{
        % E1 实验信息
        \infoTable{E1: 低温技术平台与高温超导研究}
        {2025年3月21日、2025年3月28日}
        {教学楼物理实验室A101}
        {室温26\degree}
        {黄罗琳 22344001}
        {戴鹏辉 22344016} 
        {杨舒云 22344020} 
        {丁侯凯 22344009} 
        {王凯、章嘉伟}
    }{
    \ifthenelse{\equal{\currentExperiment}{E3}}{
        % E3 实验信息
        \infoTable{E3: 光泵磁共振实验}
        {2025年5月16日}
        {A407光学实验室}
        {室温25\degree}
        {黄罗琳 22344001}
        {戴鹏辉 22344016} 
        {杨舒云 22344020} 
        {丁侯凯 22344009} 
        {刘培亮}
    }{
    \ifthenelse{\equal{\currentExperiment}{E5}}{
        % E5 实验信息
        \infoTable{E5: 双光子纠缠源研究和量子非局域性验证实验}
        {2025年3月07日}
        {A102暗室}
        {室温24\degree}
        {黄罗琳 22344001}
        {戴鹏辉 22344016} 
        {杨舒云 22344020} 
        {丁侯凯 22344009} 
        {指导老师}
    }{
    \ifthenelse{\equal{\currentExperiment}{E6}}{
        % E6 实验信息
        \infoTable{E6: 散射光成像实验}
        {2025年5月30日}
        {A102}
        {室温23\degree}
        {黄罗琳 22344001}
        {戴鹏辉 22344016} 
        {杨舒云 22344020} 
        {丁侯凯 223440XX} 
        {指导老师}
    }
    }}}
}



% 定制的节
\newcommand{\JMSection}[1]{
	\section{#1}
	\vspace{-0.7cm} % 负值缩小间距
	\noindent\textcolor{fgraygreen}{\rule{0.382\textwidth}{2pt} }
	\vspace{7pt}
}

% 个人信息表(一人一表)
\newcommand{\infoPersonal}[6]{
	\renewcommand\arraystretch{1.4}
	\begin{tabularx}{\textwidth}{|X|X|X|X|}
		\hline
		专业: & #1 &年级: & #2\\
		\hline
		学生姓名: & #3 & 学号: & #4  \\
		\hline
		实验: & #5 & 日期: & #6\\
		\hline
	\end{tabularx}
}

% 粗糙的、用于偷懒的参考文献条目(中文版)
\newcommand{\JMRef}[7]{
	\item\label{ref:#1} #2\quad#3[#4]. \emph{#5}, #6, #7.
}

% 添加了学院配套主题图标
\newcommand{\liuyin}{\mathord{\raisebox{-0.2ex}{\includegraphics[height=0.8em]{images/theme/sysuspa_icon.png}}}}

% 强调
\newcommand{\JMEmph}[1]{
	\textbf{\textcolor{flameorange}{#1}}
}

% 加了一个评分表(来自原版)
\newcommand{\scoresTable}[8]{
	\begin{table}[h!]
		\renewcommand\arraystretch{1.7}
		\begin{tabularx}{\textwidth}{
				|X|X|X|X
				|X|X|X|X|}
			\hline
			\multicolumn{2}{|c|}{预习报告} & \multicolumn{2}{|c|}{实验记录} & \multicolumn{2}{|c|}{分析讨论} & \multicolumn{2}{|c|}{总成绩} \\
			\hline
			\centering#1&\centering#2 &\centering#3 &\centering#4 &\centering#5 &\centering#6 &\centering#7 &{\centering#8} \\
			\hline
		\end{tabularx}
	\end{table}
}


% ----------------------------------------------------- 
%	下面不用看,致敬原本的流萤主题
% ----------------------------------------------------- 


% ----------------------------------------------------- 
% 我还是不太能理解,为什么他们说今晚月色很美是含蓄的表白。 
% 直到我看到朝阳下江水漾起的片片金鳞、漆黑的夜空中不甘散去的橘黄云彩、亦或者是夜宵摊子上高谈阔论掺杂着串子被炭火炙烤出油香的烟火气息,都下意识拿出手机想跟你分享。 
% 我想把自己觉得美丽的东西传递给你。 
% 今天依然天晴,我亲爱的流萤。 
\usepackage{booktabs} % 用于绘制三线表
\usepackage{siunitx}  % 用于科学计数法对齐
%---------------------------------------------------------------------
% 正文
%---------------------------------------------------------------------

\begin{document}
\experimentNumber{E3} 
\setExperimentHeader%自动生成页眉修改,自定义则如下所示
% 自定义页眉
%\fancyhead[R]{\textcolor{fgrayblue}{电阻热噪声玻尔兹曼常量测量}}
% 自定义页脚内容
%\fancyfoot[L]{\textcolor{fgrayblue}{黄罗琳,戴鹏辉,杨舒云,丁侯凯}}
%\fancyfoot[R]{\textcolor{fgrayblue}{\textbf{\thepage}}}


%给出当前实验组序号
% 实验信息列表:
% E1: 低温技术平台与高温超导研究
% E2: ECDL外腔式半导体激光器实验
% E3: 光泵磁共振实验
% E4: 晶体声光电光磁光实验
% E5: 双光子纠缠源研究和量子非局域性验证实验
% E6: 散射光成像实验
% E9: 自主研发光泵磁共振实验
% E10: 光纤特性研究与应用


% 实验报告主体部分,包含封面、前言、预习报告、实验主体内容以及附录。

    % 封面(主要修改实验参与人员和封面插图)
    % !TEX root = ../main.tex

\begin{titlepage}
	\newgeometry{margin = 0in}
	\parindent=0pt
	
	\hspace*{\fill} % 不用学院Logo就把这行注释掉
	
	\vspace{2em} % 不用学院Logo就把这行注释掉
	
	% 插图
	\begin{figure}[h!]
		\centering
		\includegraphics[width=0.8\linewidth]{images/theme/sysuspa_title}
	\end{figure}
	
	% 可选项(也就是不用学院logo,记得改注释)
%	\begin{figure}
%		\centering
%		\includegraphics[width=0.8\linewidth]{images/theme/snow_mountain_starry sky_aurora_2MB_cut}
%	\end{figure}
	
	\vspace{5em} % 不用学院Logo就把这行注释掉
	
	\begin{flushright}
		% 标题
		{ \Huge \bfseries 低温技术平台与高温超导研究} \hspace*{6em} \\[0.3cm]
		
		\rule{0.618\textwidth}{6pt} \hspace*{6em} \\[0.4cm]
		
		% 作者
		\LARGE\emph{\textbf{黄罗琳\quad 戴鹏辉\quad 杨舒云\quad 丁侯凯}} \hspace*{4.1em}	\\[0.7cm]
		
		% 地址
		\large\textbf{中山大学物理与天文学院,中国珠海市大学路 2 号,519082} \hspace*{5em}
		
	\end{flushright}
	
	% Bottom of the page
	\begin{center}
		\vfill
		
		% \Large\texttt{\textcolor{fgrayblue}{Zu messen heißt zu fragen, zu beobachten heißt zu lauschen, zu verstehen heißt zu enthüllen.}}
		%\Large\texttt{\textcolor{fgrayblue}{\kaishu 今古诸事,激荡中流,宏图待看新秀。}} % 试试这个,其实上面那个更高级
		\Large\texttt{\textcolor{fgrayblue}{\kaishu 行非凡之事,成未竟之功}} % 试试这个,其实上面那个更高级
		
		\hspace*{\fill}
		
		{\LARGE \today}
		
		\hspace*{\fill}
	\end{center}
	
\end{titlepage}
    
    % 前言(填写实验基本信息)
    % !TEX root = ../main.tex

% 封面


%---------------------------------------------------------------------
% 评分栏
\begin{table}
	\renewcommand\arraystretch{1.7}
	\begin{tabularx}{\textwidth}{
			|X|X|X|X
			|X|X|X|X|}
		\hline
		\multicolumn{2}{|c|}{Preview Report}&\multicolumn{2}{|c|}{Experimental Record}&\multicolumn{2}{|c|}{Analysis \& Discussion}&\multicolumn{2}{|c|}{\Large\textbf{Total}}\\
		\hline
		\LARGE & & \LARGE & & \LARGE & & \LARGE & \\
		\hline
	\end{tabularx}
\end{table}


%---------------------------------------------------------------------
% 信息栏
\begin{table}
	\renewcommand\arraystretch{1.7}
	\begin{tabularx}{\textwidth}{|X|X|X|X|}
		\hline
		Grade \& Major: & 2022, Physics &Group number: & A2\\
		\hline
		Student name: & 杨舒云 \& 戴鹏辉  & Student number: & 22344020 \& 22344016\\
		\hline
		Experiment time: & 2024/9/25 & Teacher's Signature: & \\
		\hline
	\end{tabularx}
\end{table}


%---------------------------------------------------------------------
% 大标题
\begin{center}
	\huge \textbf{ET2-1 \quad 蓝牙音箱的焊接和调试\\Welding and Debugging of Bluetooth Speakers}
\end{center}


%---------------------------------------------------------------------
% 注意事项
\textbf{【Precautions】}
\begin{enumerate}
	\item The lab report consists of three parts:
	\begin{enumerate}
		\item \textbf{Prview Report}: Carefully study the experimental manual before class to understand the experimental principles; familiarize yourself with the instruments, equipment, and tools needed for the experiment, and their usage; complete the pre-lab thought questions; understand the physical quantities to be measured during the experiment, and prepare the experimental record forms in advance as required (you may refer to the experiment report template and print it if needed).
		\item \textbf{Experimental Records}: Meticulously and objectively record the experimental conditions, phenomena observed during the experiment, and data collected. Experimental records should be written in ballpoint pen or fountain pen and signed (\textcolor{fred}{\textbf{Records written in pencil are considered invalid}}). \textcolor{fred}{\textbf{Keep original records, including any errors and deletions; if a correction is necessary due to an error, it must be made according to the standard procedure.}} (Records should not be entered into a computer and printed, but handwritten notes can be scanned and printed); before leaving, have the experimental teacher check and sign the records. 
		\item \textbf{Data Processing and Analysis}: Process the raw experimental data (except for experiments that focus on learning the use of instruments), analyze the reliability and reasonableness of the data; present the data and results in a standardized manner (charts and tables), including numbering and referencing the data, charts, and tables sequentially; analyze the physical phenomena (including answering the experimental thought questions, writing out the thought process, and citing data as needed according to standards); finally, draw a conclusion.
	\end{enumerate}
	\textbf{The experiment report combines the preparation report, experimental records, and data processing and analysis, along with this cover page.}
	\item Submit the \textbf{experiment report} within one week after completing each experiment (under special circumstances, no later than two weeks).
\end{enumerate}


%---------------------------------------------------------------------
% 实验安全	
%\textbf{【Safety】}	
%\begin{enumerate}
%	\item 
%\end{enumerate}	
	
	
%---------------------------------------------------------------------	
% 特别鸣谢
\textbf{【Special Note】}	

Special thanks to \textbf{Huanyu Shi}, a senior from the Class of 2019, for providing the \LaTeX \ template for this experiment report. 

\textbf{Due to the absence of an experiment number in the original template, a self-named number has been added for ease of organization on the computer.} 

Additionally, \large\textbf{\textcolor{fred}{this experiment report is being improved towards full English expression, so there may be instances of mixed Chinese and English during this transition period. We appreciate your understanding!}}
	

%---------------------------------------------------------------------	
% 目录
\clearpage
\tableofcontents		
    
    % 预习报告部分
    % !TEX root = ../main.tex

% 预习报告	
\clearpage
\setcounter{section}{0}
\section{预习报告}
\vspace{-0.7cm} % 负值缩小间距
\noindent\textcolor{fgraygreen}{\rule{0.382\textwidth}{2pt} }
\vspace{7pt}
% 【看这里】可用\ysySection{预习报告}一键实现

\subsection{实验概述}

本实验通过测量散射介质的点扩散函数和利用解卷积原理,定性探究散射光成像的机理。实验内容包括搭建几何光学成像装置、测量点扩散函数、采集未知物散斑,并利用维纳滤波等方法恢复图像,同时探讨光学记忆效应对成像恢复的影响。


\subsection{实验用具}

    \begin{table}[ht!]
    \centering
    \caption{实验用具清单}
    \label{tab:apparatus}
    \begin{tabularx}{\textwidth}{cllc}
    \toprule
    \textbf{编号} & \textbf{仪器用具名称} & \textbf{数量} & \textbf{主要参数(型号、规格等)} \\
    \midrule
    1 & 绿光LED & 1 & 
    \begin{minipage}[t]{0.6\textwidth}
    作为光源
    \end{minipage} \\
    \addlinespace
    2 & 镂空板 & 1 & 
    \begin{minipage}[t]{0.6\textwidth}
    作为成像目标物
    \end{minipage} \\
    \addlinespace
    3 & 散射片 & 4 & 
    \begin{minipage}[t]{0.6\textwidth}
    0.5°,1°,5°,10°
    \end{minipage} \\
    \addlinespace
    4 & 相机 & 1 & 
    \begin{minipage}[t]{0.6\textwidth}
    用于记录图像
    \end{minipage} \\
    \addlinespace
    5 & 镜头 & 1 & 
    \begin{minipage}[t]{0.6\textwidth}
    用于扩大视场
    \end{minipage} \\
    \bottomrule
    \end{tabularx}
\end{table}




\subsection{原理概述}
\subsubsection{点扩散函数(Point Spread Function, PSF)}
\textbf{定义:}光学系统的点扩散函数是指一个物面理想点光源通过光学系统后在像面上形成的三维光强分布。

\textbf{物理意义:}由于光的波动性(衍射效应)及光学系统的像差,物面上的理想点光源(可用 $\delta$函数描述)在像面上会形成有限大小的光斑。PSF即为光学系统对点光源的脉冲响应函数,是评价光学系统成像质量的重要指标。

\textbf{作用:}在空间平移不变的非相干成像系统中,成像过程可表示为物体与PSF的卷积运算:
\[ I_{\text{image}}(x,y) = O_{object}(x,y) \ast \text{PSF}(x,y) \]
通过测量PSF可以定量评估系统的分辨率、调制传递函数(MTF)等性能参数。在计算成像中,PSF是图像复原的关键先验知识。

\textbf{影响因素:}
\begin{itemize}
    \item 衍射极限:$\text{PSF}_{\text{diffraction}}(r) \propto \left[\frac{J_1(\pi r/\lambda F\#)}{\pi r/\lambda F\#}\right]^2$
    \item 几何像差(球差、彗差、像散等)
    \item 散射效应(介质不均匀性)
\end{itemize}



% \subsubsection{角度光学记忆效应(Angular Memory Effect)}

%     \textbf{定义:}当入射光在散射介质表面发生$\theta \leq \lambda/(2\pi L)$的角度偏转时($L$为散射介质厚度),出射散斑场将产生刚性平移而非完全重构的现象。

%     \textbf{物理机制:}源于散射介质中传播矩阵的特征值分布特性,在相关角度范围内满足:
%     \[ C(\Delta\theta) = \langle E(\theta)E^*(\theta+\Delta\theta) \rangle \approx \text{sinc}^2\left(\frac{kL\Delta\theta}{2}\right) \]
%     其中$k=2\pi/\lambda$为波数。

%     \textbf{应用:}
%     \begin{itemize}
%         \item 透过散射介质成像(记忆效应成像)
%         \item 光学相位共轭
%         \item 散斑自相关成像(需满足$\Delta\theta < \lambda/L$)
%     \end{itemize}



\subsubsection{角度光学记忆效应(Angular Memory Effect)}

    \textbf{定义:}当入射光在散射介质表面发生$\theta \leq \lambda/(2\pi L)$的角度偏转时($L$为散射介质厚度),出射散斑图样不会完全重构,而是产生刚性平移的现象。该现象说明了散射系统在小角度扰动下仍保留了部分“记忆”。

    \textbf{物理机制:}该效应源于散射介质中传播矩阵的统计特性。当入射角发生微小变化时,光在介质内部的传播路径仅发生微调,导致出射场的分布形状保持不变,仅平移而已。在数学上,出射场之间的相关性可以表示为:
    \[
    C(\Delta\theta) = \langle E(\theta)E^*(\theta+\Delta\theta) \rangle \approx \text{sinc}^2\left(\frac{kL\Delta\theta}{2}\right)
    \]
    其中,$E(\theta)$为入射角$\theta$时的出射电场,$k = 2\pi/\lambda$为波数,$\langle \cdot \rangle$表示对不同散斑点的平均。该表达式表明,当$\Delta \theta$足够小时,$C(\Delta \theta)$保持较高的相关性,即出射图样相似;而超过一定角度后,相关性迅速衰减。

    \textbf{记忆角度范围:}该效应通常在角度偏转$\Delta \theta \lesssim \lambda / L$的范围内显著。角度越小,记忆越强,平移特征越明显。

    \textbf{应用:}
    \begin{itemize}
        \item \textbf{透过散射介质成像:}利用散斑图的角度记忆特性,在不知道介质内部结构的前提下,通过偏转入射角实现目标信息的重建,即记忆效应成像。
        \item \textbf{光学相位共轭:}结合角度记忆效应,可实现散射场的逆传播与波前重构,有助于散射环境下的光学自聚焦。
        \item \textbf{散斑自相关成像:}在$\Delta\theta < \lambda/L$条件下,输出散斑的自相关函数包含目标物信息,可实现无透镜散斑成像。
    \end{itemize}




% \subsubsection{平移光学记忆效应(Translational Memory Effect)}
% \textbf{定义:}当入射光在散射介质表面发生横向位移$\Delta r \leq \lambda/\pi$时,出射散斑场产生对应平移的现象。

% \textbf{特性:}
% \begin{itemize}
%     \item 与介质厚度无关
%     \item 平移范围仅取决于波长(典型值$\sim\lambda$量级)
%     \item 满足位移-相位关系:$\Delta\phi = k\Delta r\cdot\sin\theta$
% \end{itemize}

% \textbf{应用:}
% \begin{itemize}
%     \item 散斑追踪技术
%     \item 超分辨率定位成像
%     \item 波前传感
% \end{itemize}


\subsubsection{平移光学记忆效应(Translational Memory Effect)}

    \textbf{定义:}当入射光束在空间上沿横向方向发生微小平移时(而非角度偏转),在一定条件下,出射的散斑图样仍不会发生完全重构,而是随入射光束同步平移的现象,被称为“平移光学记忆效应”。

    \textbf{物理机制:}该效应的存在依赖于散射介质中**散射路径的空间相干性**。当光束的横向位移 $\Delta x$ 小于某一相关长度(即光束在介质内部的散射路径投影仍有显著重合)时,介质对入射波前的响应仍具有较强的保真性,因此出射散斑场的结构保持不变,仅发生横向平移。理论上,其相关函数可表示为:
    \[
    C(\Delta x) = \langle E(x)E^*(x+\Delta x) \rangle \approx \text{sinc}^2\left(\frac{k \Delta x \theta}{2} \right)
    \]
    其中,$x$为入射位置,$\Delta x$为空间平移,$\theta$为散射角度,$k=2\pi/\lambda$为波数。

    \textbf{有效平移范围:}有效的平移范围受限于入射光束的空间相干长度以及介质的厚度与散射强度。在高度多次散射的厚介质中,该效应通常不如角度记忆效应明显,平移范围较小。

    \textbf{应用:}
    \begin{itemize}
        \item \textbf{光束扫描成像:}利用入射光的横向平移与出射散斑的对应平移关系,可以实现快速无透镜成像。
        \item \textbf{波前调控优化:}结合平移记忆效应与角度记忆效应,可用于提升波前优化算法对整个视场的泛化能力。
        \item \textbf{光学神经网络训练:}基于平移不变性的训练机制,可以提升系统对空间位移的鲁棒性。
    \end{itemize}






\subsubsection{解卷积算法}
成像模型表示为:
\[ I(x,y) = [O \ast \text{PSF}](x,y) + N(x,y) \]
其中$N(x,y)$为加性噪声。

\textbf{常见算法:}
\begin{enumerate}
    \item \textbf{维纳滤波}:
    \[ \hat{O}(u,v) = \left[ \frac{\text{PSF}^*(u,v)}{|\text{PSF}(u,v)|^2 + K} \right] I(u,v) \]
    其中$K = S_N(u,v)/S_O(u,v)$为噪声-信号功率比
    
    \item \textbf{Richardson-Lucy算法}(最大似然估计):
    \[ O^{(k+1)}(x,y) = O^{(k)}(x,y) \left[ \left( \frac{I}{\text{PSF}\ast O^{(k)}} \right) \ast \text{PSF}(-x,-y) \right] \]
    
    \item \textbf{稀疏约束解卷积}:
    \[ \min_O \| I - \text{PSF}\ast O \|_2^2 + \lambda \| \Psi O \|_1 \]
    $\Psi$为稀疏变换(如小波、TV正则化)
\end{enumerate}

\textbf{实验注意事项:}
\begin{itemize}
    \item PSF标定需使用亚分辨率荧光微球(直径$<\lambda/2\text{NA}$)
    \item 信噪比(SNR)需大于20dB以保证解卷积稳定性
    \item 需考虑光学系统的空间变化性(非均匀PSF)
\end{itemize}
\subsection{前思考题}
\begin{question}
	为什么只有在记忆效应范围内才能恢复成像?
\end{question}
光学记忆效应之所以存在有效范围限制,本质上是由散射介质中光传播的波矢相关性决定的。当入射光角度变化在$\theta \leq \lambda/(2\pi L)$范围内时($L$为散射介质厚度),散射介质传输矩阵的本征模式仍保持较强的空间相关性,使得出射光场与入射光场之间维持确定的线性变换关系。这一特性保证了光学系统点扩散函数(PSF)的空间平移不变性,从而使成像过程可表示为物函数与PSF的卷积运算,这是所有解卷积算法得以适用的数学基础。

一旦超出该角度范围,多重散射导致的相位积累$\Delta\phi \sim kL\theta^2$将超过$\pi$弧度,使得传输矩阵的不同本征模式之间完全解耦。此时散斑图样会发生本质性改变而不仅是刚性平移,破坏了PSF的空间不变性,卷积模型不再成立。此外,从信息论角度看,超出记忆效应范围后,散射过程引入的熵增使系统信道容量急剧下降,导致物体信息被不可逆地湮没在噪声中。实际成像还受光学系统数值孔径的限制,有效视场$FOV \approx \lambda/(2\text{NA})$需与记忆效应范围匹配,才能保证在可恢复区域内既有足够的光学信息量,又能维持准确的卷积关系。
\begin{question}
	请简述卷积定理。
\end{question}

    \textbf{卷积定理}指出,两个函数在时域(或空域)中的卷积等价于它们在频域中的乘积,反之亦然。具体表述为:

设$f(x)$和$g(x)$是两个可积函数,其傅里叶变换分别为$F(\omega)$和$G(\omega)$,则它们的卷积$h(x) = (f * g)(x)$的傅里叶变换$H(\omega)$等于各自傅里叶变换的乘积:
$$
H(\omega) = F(\omega) \cdot G(\omega)
$$
反之,两函数在时域中的乘积的傅里叶变换等于它们各自傅里叶变换的卷积:
$$
\mathcal{F}\{f(x) \cdot g(x)\} = F(\omega) * G(\omega)
$$

数学体现:
    \begin{itemize}
        \item 空域卷积:$I(x,y) = O(x,y) \ast \text{PSF}(x,y)$  
        \item 频域乘积:$\mathcal{F}\{I\} = \mathcal{F}\{O\} \cdot \mathcal{F}\{\text{PSF}\}$  
    \end{itemize}
    


    
    % 主要实验内容
    % !TEX root = ../main.tex

% 报告主体

%%%
%---------------------------------------------------------------------
% 实验
\clearpage
\JMSection{实验过程}
\subsection{实验前期准备}

\subsubsection{接线准备}

由于实验台的选择,本小组主要完成对于超导转变温度的测量,这里就需要对于样品的电阻进行测量,对于测量电路的接线,主要通过如下所示的接线图来进行:
\begin{figure}[{H}]
	\centering
	\includegraphics[width=0.45\linewidth]{pre3.png}
	\caption{交流四引线法}
	\label{}
\end{figure}

实际操作过程中,对于实验线路操作,经过了很多尝试,由于对于接线过程中相关器件的内部结构的不熟悉,经历了很多次纠错的经过,最终接线如下:
\begin{figure}[{H}]
	\centering
	\includegraphics[width=0.7\linewidth]{body1.jpg}
	\caption{实际接线图}
	\label{}
\end{figure}

\subsubsection{温度准备}

实验主要通过液氮降温,对于理论的转变温度来说,液氮理论上已经满足实验所需的温度条件,\JMEmph{但是实际操作过程中,我们发现,实际上根据Pt1000给出的数值再与讲义后的对照表进行对照,发现温度与控温器给出的温度相差很大,这一点会在分析讨论部分给出本小组的意见。}

除了常规液氮降温,在实验过程中,\JMEmph{在老师的启发下,本小组将恒温器液氮部分的出气口连接到一个抽气泵,通过加快液氮蒸发的速度,从而达到更低温度的效果,但是实际实验结果来看,效果并没有达到预期。}

\subsubsection{LabVIEW准备}

实验软件的准备也同样出现了很多问题,基本上也属于摸索阶段,对于接口的试错导致不得不一次又一次进行重启软件,最后得出了如下接口对应关系:

\begin{itemize}
	\item \textbf{控温仪 TC202}  $\rightarrow$ COM7
	\item \textbf{3058E} $\rightarrow$ USB0::0X1
	\item \textbf{电磁铁电源} $\rightarrow$ COM1
	\item \textbf{锁相放大器} $\rightarrow$ COM3
\end{itemize}



\subsection{转变温度测量}
温度区间取80k至110k(绘图时对于数据进行了裁剪),功率采用15\%,升温降温分别测量,最终得出的实验数据绘图后显示为:
\begin{figure}[{H}]
	\centering
	\includegraphics[width=0.7\linewidth]{body2.png}
	\caption{升温与降温转变温度}
	\label{}
\end{figure}

\subsection{外加磁场对样品转变温度影响}
\subsubsection{实验过程}
本实验是选择了讲义中给出的众多实验问题中其中一个:

\highlight{外磁场或外加电流如何影响超导转变时的电阻随温度($T_c$)的变化?}

\JMEmph{相关实验方案已经在本实验报告预习报告部分给出了,本方案于实验的第二周进行。}

其中与讲义不同的是,讲义中给出需采用电磁铁电源中的磁场模式(FIELD),但是在本实验中采用的是电流模式,故通过特斯拉计测量出如下电流与磁场对应关系:

\begin{table}[h!]
	\centering
	\renewcommand{\arraystretch}{1.5} % 调整行高
	\caption{电流与磁场的关系}
	\label{tab:current_magnetic_field}
	\begin{tabular}{|c|c|c|c|c|c|c|c|c|}
		\hline
		\textbf{电流/A} & 1 & 2 & 3 & 4 & 5 & 6 & 7 & 8\\ \hline
		\textbf{磁场/KGS} & 0.6732 & 1.2414 & 1.8436 & 2.4955 & 3.0319 & 3.6071 & 4.1647 & 4.7109\\ \hline
	\end{tabular}
\end{table}


最终数据输出结果如下,相关数据讨论见分析讨论部分:
\begin{figure}[{H}]
	\centering
	\includegraphics[width=0.8\linewidth]{body3.jpg}
	\caption{不同线圈电流下,样品在降温过程的电阻变化}
	\label{11}
\end{figure}
\begin{figure}[{H}]
	\centering
	\includegraphics[width=0.8\linewidth]{body4.jpg}
	\caption{不同线圈电流下,样品在升温过程的电阻变化}
	\label{22}
\end{figure}


\subsubsection{初步讨论}

从实验结果来看,并不能一个能够分析出规律的结果。

在实验现场当我们发现这个问题后,经过与老师讨论,通过的如下一系列操作优化实验:
\begin{itemize}
	\item 减小升温功率,之前为保证升温功率能够使得恒温器能够达到目标温度,采用了15\%挡位,但是考虑到可能会导致升温速率过快,器件传导不及时等问题,降低了升温功率为1\%,但是随之而来的问题就变为,当温度在84K左右时,已经无法升温了,不得不再次调回15\%挡位,实际结果并没有区别。
	\item 调整液氮挡位,由于考虑到,液氮在器件中可能冷却能力不足,通过调整器件挡杆来调整液氮,但是实际上也收效甚微。
	\item 选择读取Pt1000给出的电阻值来进行温度换算,但是与实际温度对照表进行对照后,发现,大相径庭,如果选择Pt1000的数据作为实际测量温度的话,那么超导转变温度已经与实际测量温度相差甚远。
\end{itemize}

\JMEmph{尝试了众多优化方案后,依旧无法得出一个具有规律的实验结果,最终本小组决定采用最开始的实验数据进行分析,尽管没有完成实验预期,但是在分析讨论部分,我会给出个人对于实验优化方向的理解与讨论。}



\clearpage

%---------------------------------------------------------------------
% 分析与讨论
\JMSection{分析与讨论}

\subsection{数据分析}

\subsubsection{超导讨论}
对于判断待测样品是否为超导体,通常需要基于超导体的两个基本特性来进行判断。
这两个基本特性是:

\begin{enumerate}
    \item \textbf{零电阻性}:超导体在其临界温度以下会表现出零电阻性。当样品温度降到其超导转变温度(通常称为临界温度,$T_c$)以下时,其电阻将突然降为零,并且在整个超导状态下维持零电阻。通过测量电阻与温度的关系,可以确认这一点。如果在某一温度点电阻骤降至零,并在后续保持不变,则样品有可能是超导体。
    
    \item \textbf{完全抗磁性(迈斯纳效应)}:超导体具有完全抗磁性,也就是说,当超导体进入超导状态时,它会排斥磁场,表现为磁通量完全被排除在其内部。这种现象称为迈斯纳效应。实验上,通常通过测量样品在低于临界温度时的磁响应来确认这一点。如果在低于临界温度时样品的磁场强度显著减弱或消失,则说明样品可能具有迈斯纳效应,表明它可能是超导体。
\end{enumerate}

因此,要判断待测样品是否超导,实验中需要:
\begin{itemize}
    \item \textbf{测量电阻}:观察样品的电阻随温度变化的情况,确认是否存在零电阻的转变。
    \item \textbf{测量磁性}:使用磁场探测设备测量样品在临界温度以下的磁场响应,确认是否存在迈斯纳效应。
\end{itemize}

\subsubsection{超导转变温度}
\begin{figure}[H]
	\centering
	\includegraphics[width=0.7\linewidth]{day1.png}
	\caption{升温与降温转变温度}
	\label{fig:day1}
\end{figure}

第一周实验中,我们的重点是熟悉仪器,同时测量了一组数据,即“样本在温度升高和降低的条件下的电阻值”。数据绘制的图像如上图所示。

可以发现,\JMEmph{在升温过程中,温度在93.21K时,电阻值发生明显的转变,在94.35K时,电阻值转变开始趋于平缓;在降温过程中,温度在86.57K时,电阻值发生明显转变,在85.28K时,电阻值转变开始趋于平缓。}

上面的实验数据说明,该样品在\highlight{85K-94K存在超导转变现象。}

值得注意的是,升温过程的转变温度与降温过程的转变温度相差有\highlight{8K},显然在温度的测量中应该存在误差,这在后面再进行更详细的讨论。

\subsubsection{磁场对于超导转变温度影响}

在第二周的实验中,我们探究了被测样品在不同大小的外加磁场对样品转变温度的影响。实验数据的绘制图像如下所示:



我们将电阻值转变的部分截取出来,并标记出电阻值开始明显变化的点,如下所示。

\begin{figure}[H]
	\centering
	\includegraphics[width=0.7\linewidth]{1.jpg}
	\caption{电阻值转变开始变化的点}
	\label{fig:1}
\end{figure}

\begin{figure}[H]
	\centering
	\includegraphics[width=0.7\linewidth]{2.jpg}
	\caption{电阻值转变开始变化的点}
	\label{fig:2}
\end{figure}

\begin{figure}[H]
	\centering
	\begin{minipage}{0.4\linewidth}
		\centering
		\includegraphics[width=\linewidth]{8.png}
		\caption{降温过程的转变温度与线圈电流的关系}
		\label{fig:8}
	\end{minipage}
	\hspace{0.04\linewidth}
	\begin{minipage}{0.4\linewidth}
		\centering
		\includegraphics[width=\linewidth]{7.png}
		\caption{升温过程的转变温度与线圈电流的关系}
		\label{fig:7}
	\end{minipage}
\end{figure}


图\ref{fig:8}和\ref{fig:7}分别展示的是样品在降温和升温过程的转变温度与线圈电流的关系。

从图中可以看出,该样品的转变温度变化对外磁场并不敏感,转变温度的上下波动更多的是测量误差。
\subsubsection{理论建模}
\JMEmph{但是实际上,根据理论模型,本实验的操作会给出如下的预期实验结果:}

超导体的超导转变温度(临界温度 $T_c$)受到外加磁场的影响,外磁场的作用主要表现为两个方面:
\begin{itemize}
    \item \textbf{磁场对超导电子配对的破坏作用}:超导现象的本质是库珀对(Cooper pairs)的形成,这些配对的电子在无外磁场时通过相互吸引形成超导态。当施加外磁场时,磁场会破坏这些库珀对,因为磁场会破坏电子对的相干性。
    \item \textbf{临界磁场的存在}:每种超导材料都有一个特定的临界磁场($H_{c}$),超过这个临界磁场,超导体就会转变为正常态。外加的磁场会影响超导体的临界温度,因此,随着外磁场的增大,超导体的临界温度会降低。
\end{itemize}
\begin{figure}[{H}]
	\centering
	\includegraphics[width=0.6\linewidth]{9.png}
	\caption{超导临界温度、临界电流密度与临界磁场之间的关系}
	\label{}
\end{figure}

阿尔希博格-吉尔伯特模型

在低温下,超导体的临界温度 $T_c$ 与外加磁场 $H$ 之间的关系可以用经验公式来描述。对于许多超导体(如高温超导体 YBa$_2$Cu$_3$O$_{7-\delta}$),这种关系可以近似表示为:
\[
T_c(H) = T_c(0) \left(1 - \frac{H}{H_{c2}}\right)
\]
其中:
\begin{itemize}
    \item $T_c(H)$ 是在磁场 $H$ 下的超导临界温度;
    \item $T_c(0)$ 是在零磁场下的超导临界温度;
    \item $H_{c2}$ 是超导体的二级临界磁场(即磁场强度达到 $H_{c2}$ 时,超导状态完全消失)。
\end{itemize}

该公式表明,随着外磁场的增大,临界温度 $T_c$ 会线性下降,直到在临界磁场 $H_{c2}$ 处,临界温度完全为零,超导体失去超导性质。

此外,外磁场不仅影响超导临界温度,还会影响超导转变的过程。随着外磁场的增大,超导体的超导转变过程变得更加缓慢。这是因为:
\begin{itemize}
    \item \textbf{库珀对的破坏}:外磁场破坏了库珀对的形成,因此超导转变的过程变得更加平缓。
    \item \textbf{涡旋动态的引入}:在高磁场下,超导体内部会形成涡旋(vortices)。涡旋是磁场线通过超导体的区域,其中超导电流不能流动,从而增加了磁场下的能量损失,并使得超导转变变得更加缓慢。
\end{itemize}

\JMEmph{根据上述理论,当外磁场逐渐增大时,超导临界温度会下降,并且转变过程变得更加缓慢,但是根据实际的实验数据并不能得出如此结论。}

\subsection{实验进一步讨论}
\subsubsection{温度的测量}

理论上,我们预期的结果应该是,不论是升温过程还是降温过程,都有相同的转变温度;但是实际实验的测量结果显示,升温过程的转变温度为94K左右,降温过程的转变温度为86K左右,相差8K,这已经超出了简单的“测量误差”的接受范围。

\JMEmph{一个可能的原因是温度传感器的温度并未真实的反应样品的温度,而是存在一个滞后。}在我们的实验中,升温与降温采用动态法测量,温度变化速率不同,导致样品与温度计的温度响应存在差异。\JMEmph{升温时,温度计显示温度比样品实际温度高,测得转变温度偏高;降温时,温度计显示温度比样品实际温度低,测得转变温度偏低。}

一个改进的方向是,降低温度变化的速率,让被测样品与温度传感器之间达到热平衡,以此测得更准确的温度。

\subsubsection{电阻的测量}

理论上,我们应当预期在超导转变前后,样品具有相同的电阻值;但是测量结果显示,在反复测量的过程中,测量得到的电阻值存在较大的波动。

可能的原因是,我们所使用的压控电流源并不稳定,这会导致锁相放大器所测量得到的电压值也不稳定,造成较大的波动。

另一个可能的原因是,样品中可能存在温度梯度,会产生热电效应,导致额外的电压或电流,从而影响测量结果。



\subsection{总结讨论}

\begin{ubox}{实验结果}
	\quad \quad 总的来说,实验并没有得出一个合适的结果,尽管我们找到了超导转变温度带来的陡升和陡降,但是从结果来看,我们无法分析出符合理论模型的实验结果,实验数据分析部分也讨论了关于理论模型的预期结果,与我们的相差很大。
\end{ubox}
\begin{ubox}{优化方向}
	\quad \quad 我曾经在某研究所参与过一段时间的相关实验,其中某一环节需要使用到低温环境,所采用的同样是液氮降温,采用的与本实验中类似的恒温装置——一个低温杜瓦,根据当时的实验结果的数据来看,的确做到了正常的工作状态。
    \begin{figure}[H]
        \centering
        \begin{minipage}{0.3\textwidth} % 设置图片宽度
            \centering
            \includegraphics[width=\linewidth]{body5.jpg}
            \caption{低温杜瓦}
        \end{minipage}%
        \hfill
        \begin{minipage}{0.55\textwidth} % 设置文字的宽度
			\quad \quad 受此启发,对于实验中使用的漏热式液氮恒温器结构,我觉得一个优化方向是,提高实验装置的集成,通过设计紧凑型的低温杜瓦,能够将温控设备与样品测量系统更紧密地结合,减少热损失,提高实验的响应速度。\JMEmph{但是可能受限于样品本身的制作,以及具体仪器的尺寸问题。}
            
			\quad \quad 此外,增加集成的另一个方向是,实验仪器中,我们连接了大量的接线进行实验,比如为了分出锁相放大器的A和B,我们连接了很繁琐的电路将其分开,这些接线带来的误差也可能对于实验结果产生影响,数据存在抖动(有些抖动极大),所以对于实验电路的集成也是优化方向之一。

            \quad \quad 最后,实验的控温算法也是优化方向之一,首先功率只有三挡,并且挡位与仪器实际的升温降温的环境并不匹配,这就导致了,功率低升不上去,功率高升得太快,从而导致了,温度测量延迟等问题;由于控温采用的时PID控温,参数的设定是否符合实验仪器本身,也可能导致实际控温效果较差,导致了最后实验结果的失败。
        \end{minipage}
    \end{figure}
\end{ubox}





    
    % 附录(包含补充材料、个人签名、实验台整理情况、原始数据记录、实验附表、代码及参考文献)
    
\clearpage
\JMSection{附录}

\subsection{补充内容}

\begin{enumerate}
	\item 本实验,实验部分主要由杨舒云、丁侯凯负责,报告部分主要由黄罗琳、戴鹏辉负责,小组成员分工明确,工作任务平均。
	
	\item 感谢刘老师在实验中详细为我们解答每一个问题,让我们能够按时完成实验,感谢您!
	\item 相关源文件(\LaTeX)已上传Github,相关库包含实验报告内容,如需要查看可联系我进行查阅。
\end{enumerate}
\quad \large \textbf{感谢您对于此篇实验报告的阅读与批改,祝您工作顺利!}

\subsection{原始数据与桌面}
 \begin{figure}[H]
    \centering
    \begin{minipage}[b]{0.45\linewidth}
        \centering
        \includegraphics[width=\linewidth]{dt1.jpg}
        \caption{原始数据1}

    \end{minipage}
    \hfill
    \begin{minipage}[b]{0.45\linewidth}
        \centering
        \includegraphics[width=\linewidth]{dt2.jpg}
        \caption{原始数据2}
    
    \end{minipage}

\end{figure}
 \begin{figure}[H]
    \centering
    \begin{minipage}[b]{0.45\linewidth}
        \centering
        \includegraphics[width=\linewidth]{dt3.jpg}
        \caption{原始数据3}

    \end{minipage}
    \hfill
    \begin{minipage}[b]{0.45\linewidth}
        \centering
        \includegraphics[width=\linewidth]{dt4.jpg}
        \caption{原始数据4}

    \end{minipage}


\end{figure}
\subsection{个人签名}



\begin{figure}[H]
    \centering
    \begin{minipage}{0.45\textwidth} % 第一个图片的宽度
        \centering
        \includegraphics[width=\linewidth]{签字.jpg}
        \caption{个人签名:黄罗琳}
        \label{}
    \end{minipage}%
    \hfill
    \begin{minipage}{0.45\textwidth} % 第二个图片的宽度
        \centering
        \includegraphics[width=\linewidth]{dph.jpg}
        \caption{个人签名:戴鹏辉}
        \label{}
    \end{minipage}
\end{figure}

\begin{figure}[H]
    \centering
    % 第一个图片
    \begin{minipage}{0.45\textwidth} % 设置宽度为 0.45,留一些空间
        \centering
        \includegraphics[width=\linewidth]{name.png}
        \caption{个人签名:杨舒云}
        \label{fig:name}
    \end{minipage}
    \hfill % 在两个图片之间添加水平间隔
    % 第二个图片
    \begin{minipage}{0.45\textwidth} % 设置宽度为 0.45,留一些空间
        \centering
        \includegraphics[width=\linewidth]{dhk.png}
        \caption{个人签名:丁侯凯}
        \label{fig:dhk}
    \end{minipage}
\end{figure}
    
\end{document}
