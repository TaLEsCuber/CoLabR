% !TEX root = ../main.tex

% 预习报告	
\clearpage
\setcounter{section}{0}
\section{预习报告}
\vspace{-0.7cm} % 负值缩小间距
\noindent\textcolor{fgraygreen}{\rule{0.382\textwidth}{2pt} }
\vspace{7pt}
% 【看这里】可用\ysySection{预习报告}一键实现

\subsection{实验目的}
\begin{itemize}
    \item 通过实验加深对原子超精细能级结构、光跃迁及磁共振现象的了解。
    \item 掌握光泵磁共振的光抽运原理和光探测手段,观察光磁共振信号。
    \item 掌握精确测量铷原子 $ g_F $ 因子和弱磁场大小的实验方法。
    \item 了解吸收池中原子碰撞弛豫过程及环境磁场对吸收信号的影响。
    \item 学会利用光泵磁共振技术开展探索性实验研究。
\end{itemize}
\subsection{仪器用具}

\begin{table}[ht!]
    \centering
    \caption{光泵磁共振实验仪器清单}
    \label{tab:apparatus}
    \begin{tabularx}{\textwidth}{cllc}
    \toprule
    \textbf{编号} & \textbf{仪器用具名称} & \textbf{数量} & \textbf{主要参数(型号,测量范围,精度等)} \\
    \midrule
    1 & 直流电源 & 1 & 
    \begin{minipage}[t]{0.6\textwidth}
    第1路:0$\sim$1\,A可调稳流电源;\\
    第2路:0$\sim$0.2\,A可调稳流电源;\\
    第3路和第4路:24\,V/2\,A、20\,V/0.5\,A稳压电源
    \end{minipage} \\
    \addlinespace
    2 & 辅助源 & 1 & 
    \begin{minipage}[t]{0.6\textwidth}
    提供三角波、方波扫场信号及温度控制电路,\\
    设有"外接扫描"插座,可接示波器扫描输出
    \end{minipage} \\
    \addlinespace
    3 & 射频信号发生器 & 1 & 频率范围100\,kHz$\sim$1\,MHz,输出功率在50\,Ω负载上$\geq$0.5\,W  \\
    \addlinespace
    4 & 示波器 & 1 & -- \\
    \addlinespace
    5 & 铷光谱灯 & 1 & -- \\
    \addlinespace
    6 & 干涉滤波片 & 1 & -- \\
    \addlinespace
    7 & 准直透镜 & 1 & -- \\
    \addlinespace
    8 & 偏振片 & 1 & -- \\
    \addlinespace
    9 & 1/4玻片 & 1 & -- \\
    \bottomrule
    \end{tabularx}
    \end{table}
    
\subsection{原理概述}


\subsubsection{光泵磁共振原理}
光泵磁共振原理可以简述成下面三个过程:
\begin{enumerate}
    \item 首先通过光抽运过程使原子吸收某种特定的光造成能级原子数的分布偏离热平衡条件下的玻尔兹曼分布。
    \item 同时作用射频电磁场后原子超精细结构塞曼子能级间发生磁共振。
    \item 最后利用光探测方法探测原子对入射光的吸收,从而获得光泵磁共振的信号。
\end{enumerate}

\subsubsection{铷原子能级结构}
本实验的研究对象选用天然的气态碱金属铷(Rb)原子。在耦合中,原子价电子总角动量与原子总磁矩的关系为
$$
\boldsymbol{\mu} = -g_F \mu_B \mathbf{F}/\hbar
$$
其中,$g_F$为朗德因子。$^{85}$Rb的原子也具有自旋和磁矩,相应的自旋量子数分别为$I=5/2$和$g_I$。原子的核磁矩与电子总磁矩相互作用,产生$I$-$J$耦合,使能级进一步分裂,形成超精细结构能级。

设原子核的角动量为$\mathbf{I}$,则原子的总角动量为
$$
\mathbf{F} = \mathbf{J} + \mathbf{I}
$$
$F$为耦合后的总量子数。$^{85}$Rb基态有$J=1/2$,故取$F=2$和$F=3$。$^{87}$Rb的基态有$J=1/2$,故取$F=1$和$F=2$两个值。

原子总磁矩与总角动量的关系为
$$
\boldsymbol{\mu}_F = -g_F \mu_B \mathbf{F}/\hbar
$$
其中,朗德因子
$$
g_F = g_J \frac{F(F+1) + J(J+1) - I(I+1)}{2F(F+1)}
$$

在磁场中,原子的超精细能级产生塞曼分裂(弱场时为反常塞曼效应)。根据空间量子化原理,原子总角动量在磁场方向的投影为$m_F\hbar$,磁量子数可能值为$-F, -F+1, \dots, F$,即分裂成$2F+1$个能量间距基本相等的塞曼子能级。

在外磁场中铷原子的哈密顿量可以写成
$$
\mathcal{H} = A \mathbf{I} \cdot \mathbf{J} - \boldsymbol{\mu} \cdot \mathbf{B}
$$
其中,$A$为磁偶极子相互作用常数。

\subsubsection{光抽运过程}
实验中主要观测的是发生基态塞曼子能级之间的射频磁共振。在热平衡状态下,粒子服从玻尔兹曼分布
$$
N_i = N_0 \mathrm{e}^{-E_i/k_B T}
$$
其中,$k_B$为玻尔兹曼常数。

% 在室温下,$k_B T$与$\hbar \omega$相比较,$k_B T \gg \hbar \omega$。所以$N_i \approx N_0$,即铷原子大部分粒子都处于基态上。但对于超精细结构塞曼子能级,相邻的子能级间的能级间隔$\Delta E \ll k_B T$。所以,可以认为$N_i \approx N_j$。

光抽运通过选择性激发,使高能级的粒子数超过低能级,这是违反玻尔兹曼分布的,但是增大粒子在超精细结构塞曼子能级间布局数之差,这种不均匀分布称为偏极化,有利于为本实验观测射频磁共振创造条件。

\subsubsection{弛豫过程}
在热平衡状态下,基态各子能级上的粒子数遵从玻耳兹曼分布 ($N_i = N_0 \mathrm{e}^{-E_i/k_B T}$)。由于各子能级的能量差极小,可以近似认为各能级粒子数相等。光抽运破坏这种平衡,产生显著的非热平衡分布。系统从非平衡趋向平衡的过程称为弛豫,主要包括三种机制:
\begin{enumerate}
    \item 铷原子与容器壁碰撞(主导弛豫途径,原子面密度约$10^{14}\,\text{cm}^{-2}$)使原子回归热平衡分布;
    \item 铷原子间自旋-自旋交换弛豫(外磁场为零时消除塞曼子能级偏极化);
    \item 铷原子与缓冲气体(如$\mathrm{N}_2$)碰撞,其磁矩扰动可忽略但能大幅减少器壁碰撞概率(降低6个数量级)。缓冲气体还通过无辐射跃迁使激发态原子如$^{85}$Rb等概率返回基态,反而促进特定子能级($^{85}$Rb的$F=3, m_F=+2$,$^{87}$Rb的$m_F=+3$)的粒子数积累。
\end{enumerate}
光抽运建立的粒子数差比玻耳兹曼分布高数个数量级,这是原子气室实验设计的重要理论基础。

\subsubsection{塞曼能级间的磁共振}
对于$^{87}$Rb,在光抽运作用下,大量原子聚集到基态$F=2, m_F=+2$子能级上,实现粒子数反转,直至偏极化达到饱和。在弱的外磁场$B_0$中相邻超精细塞曼子能级能量之差为
$$
\Delta E = g_F \mu_B B_0
$$
如果在垂直于外磁场的方向上加一频率为$\nu$的射频场,当满足磁共振条件时,即
$$
h\nu = \Delta E = g_F \mu_B B_0
$$
在基态塞曼子能级间将发生射频受激辐射,$m_F = +2$子能级上的原子感应跃迁到$ m_F = +1, m_F = 0$等能量更低的能级上。同时由于抽运光的连续照射,处于基态子能级上的粒子又将被抽运到$m_F = +2$子能级上。感应跃迁与光抽运将达到一个新的动态平衡。在产生磁共振时,基态各子能级上的粒子数大于不共振时,因此对$D_1\sigma^+$光的吸收增大。光跃迁速率比磁共振跃迁速率大好几个数量级,所以光抽运过程与磁共振过程可以连续的进行下去。对于$^{85}$Rb也有类似的情况,只是光将$^{85}$Rb抽运到基态$F=3, m_F=+3$子能级上。

本次实验过程中,使用磁场固定,频率改变的扫频法来实现磁共振。使垂直于外磁场的方向上加一频率为$\nu$的射频场,当满足磁共振条件时,即
$$
h\nu = g_F \mu_B B_0
$$
从而可以观察到铷原子对$D_1\sigma^+$光的周期性吸收现象。

\subsubsection{光探测}
磁共振相伴随有对光吸收的变化,因此测光强的变化即可得到磁共振的信号,这就实现了磁共振的光探测。所以射到样品上的光同时起了抽运与探测两个作用。将一个低频射频光子(1MHz)转换成了一个高频光频光子($10^8$ MHz),使得信号功率提高了7个数量级,探测灵敏度大为提高。


\subsection{实验前思考题}


\begin{question}
光泵磁共振与核磁共振、电子顺磁共振主要区别是什么?
\end{question}

\begin{itemize}
    \item 相邻两塞曼子能级间粒子数的差别,不是由玻尔兹曼分布所决定的,而是利用光抽运的方法使相邻两能级间粒子数之差增加几个数量级;
    \item 磁共振不是通过磁偶极子跃迁所辐射或吸收的功率大小来检测,而是测量磁偶极子共振跃迁后伴随的通过样品吸收池的偏振光强度的变化。由于光量子(约 $10^8$ MHz)比射频量子(约 1- 10 MHz)的能量高 7- 8 个数量级,这样使得探测的信号功率也提高了 7-8 个数量级。通过比较可以看出,光泵磁共振方法既保持了磁共振的高分辨率,同时又提高了探测信号的灵敏度。
\end{itemize}

\begin{question}
利用理论公式计算 $^{87}$Rb 和 $^{85}$Rb 基态超精细结构能级的值,并分析在实验观测中如何区别这两种同位素的磁共振信号?
\end{question}


    \begin{itemize}
    
        \item 计算\(g_F\)值:

            \begin{itemize}
                \item Rb 的两个稳定同位素 $^{87}$Rb 和 $^{85}$Rb 均处于基态 $5S_{1/2}$,对应电子态的角动量量子数为 $L = 0$,自旋量子数为 $S = \frac{1}{2}$,因而 $J = \frac{1}{2}$。其核自旋分别为:
                    \begin{itemize}
                        \item $^{87}$Rb:$I = \frac{3}{2}$,对应总角动量 $F = I \pm J = 1, 2$;
                        \item $^{85}$Rb:$I = \frac{5}{2}$,对应总角动量 $F = 2, 3$。
                    \end{itemize}
                
                \item 首先计算电子态朗德因子 $g_J$:
                    \[
                    g_J = 1 + \frac{J(J+1) + S(S+1) - L(L+1)}{2J(J+1)}
                    \]
                    对于 $J = \frac{1}{2}, S = \frac{1}{2}, L = 0$,代入得:
                    \[
                    g_J = 1 + \frac{\frac{1}{2}(\frac{1}{2}+1) + \frac{1}{2}(\frac{1}{2}+1)}{2 \cdot \frac{1}{2}(\frac{1}{2}+1)} = 1 + \frac{\frac{3}{4} + \frac{3}{4}}{\frac{3}{2}} = 1 + 1 = 2
                    \]
                
                \item 接着利用公式计算超精细结构能级的朗德因子 $g_F$:
                    \[
                    g_F = g_J \cdot \frac{F(F+1) + J(J+1) - I(I+1)}{2F(F+1)}
                    \]
                
                \item 对于 $^{87}$Rb($I = \frac{3}{2}, J = \frac{1}{2}, g_J = 2$):
                    \begin{itemize}
                        \item $F = 1$:
                        \[
                        g_F = 2 \cdot \frac{1(1+1) + \frac{1}{2}(\frac{1}{2}+1) - \frac{3}{2}(\frac{3}{2}+1)}{2 \cdot 1(1+1)} = 2 \cdot \frac{2 + \frac{3}{4} - \frac{15}{4}}{4} = 2 \cdot \left( -\frac{10}{16} \right) = -\frac{5}{4}
                        \]
                        \item $F = 2$:
                        \[
                        g_F = 2 \cdot \frac{2(2+1) + \frac{1}{2}(\frac{1}{2}+1) - \frac{3}{2}(\frac{3}{2}+1)}{2 \cdot 2(2+1)} = 2 \cdot \frac{6 + \frac{3}{4} - \frac{15}{4}}{12} = 2 \cdot \frac{12}{48} = \frac{1}{2}
                        \]
                    \end{itemize}

                \item 对于 $^{85}$Rb($I = \frac{5}{2}, J = \frac{1}{2}, g_J = 2$):
                    \begin{itemize}
                        \item $F = 2$:
                        \[
                        g_F = 2 \cdot \frac{2(2+1) + \frac{1}{2}(\frac{1}{2}+1) - \frac{5}{2}(\frac{5}{2}+1)}{2 \cdot 2(2+1)} = 2 \cdot \frac{6 + \frac{3}{4} - \frac{35}{4}}{12} = 2 \cdot \left( -\frac{13}{24} \right) = -\frac{13}{12}
                        \]
                        \item $F = 3$:
                        \[
                        g_F = 2 \cdot \frac{3(3+1) + \frac{1}{2}(\frac{1}{2}+1) - \frac{5}{2}(\frac{5}{2}+1)}{2 \cdot 3(3+1)} = 2 \cdot \frac{12 + \frac{3}{4} - \frac{35}{4}}{24} = 2 \cdot \frac{16}{96} = \frac{1}{3}
                        \]
                    \end{itemize}

                \item 计算结果总结如下:
                    \[
                    \begin{aligned}
                    ^{87}\mathrm{Rb}: \quad & g_{F=1} = -\frac{5}{4}, \quad g_{F=2} = \frac{1}{2} \\
                    ^{85}\mathrm{Rb}: \quad & g_{F=2} = -\frac{13}{12}, \quad g_{F=3} = \frac{1}{3}
                    \end{aligned}
                    \]
            \end{itemize}


            而我们的磁共振过程主要就是发生在 $^{87}$Rb 的基态 $F = 2$ 的超精细能级上,对应的朗德因子$g_{F = 2}(^{87}\text{Rb}) = \frac{1}{2}$,以及$^{85}$Rb 的基态 $F = 3$ 的超精细能级上,对应的朗德因子$g_{F = 3}(^{85}\text{Rb}) = \frac{1}{3}$。

        \item 区别这两种同位素的磁共振信号:
            \begin{itemize}
                \item 共振频率不同:

                    磁共振频率由以下公式得到
                    \[
                        \nu=\frac{g_F\mu_B B_0}{h}
                    \]

                    其中\(\mu_B\)是玻尔磁子,\(B_0\)是恒定磁场,\(h\)是普朗克常数。

                    由于\(^{87}\)Rb与\(^{85}\)Rb的\(g_F\)值不同,它们的共振频率也不同。在相同磁场\(B_0\)下,\(^{87}\)Rb的共振频率高于\(^{85}\)Rb。

                \item 超精细结构分裂不同:\(^{87}\)Rb的超精细分裂较大,而\(^{85}\)Rb的超精细分裂较小,在实验中可以通过调节射频场的频率分别观测到两种同位素的共振信号。

                \item 信号强度不同:由于\(^{87}\)Rb和\(^{85}\)Rb的自然丰度不同,它们的信号强度也会有所不同。 
            \end{itemize}

    \end{itemize}



\begin{question}
当用铷的 D$_1$ 光或线偏振光照射处于磁场中的铷原子时,能否发生光抽运效应?若入射光为椭圆偏振光时又如何?
\end{question}

当入射光为 $\pi$ 光时,铷原子对光有强的吸收,但无光抽运效应;当入射光为椭圆偏振光(不等量的 $\sigma^+$ 和 $\sigma^-$ 的混合)时,光抽运效应较圆偏振光小。

\begin{question}
实验观测铷的磁共振信号时需要提供哪几个磁场?有什么要求?各起什么作用?
\end{question}
\begin{itemize}
    \item[\textbf{(1)}] \textbf{竖直和水平磁场}
    \begin{itemize}
        \item \textbf{作用:} 
        \begin{itemize}
            \item 消除地球磁场及其他环境杂散磁场对铷原子的影响,确保铷原子能级的塞曼分裂仅由实验所需的磁场控制。
            \item 为后续磁场的叠加提供“零磁场”的基准环境。
        \end{itemize}
        \item \textbf{要求:}
        \begin{itemize}
            \item 补偿精度高:通过三维亥姆霍兹线圈(或梯度补偿线圈)产生与地磁场大小相等、方向相反的磁场,需借助磁力计校准至剩余磁场小于 $10^{-9}$ 量级。
            \item 稳定性好:补偿后的剩余磁场需高度稳定,避免温度漂移或电流波动引起的干扰。
        \end{itemize}
    \end{itemize}

    \item[\textbf{(2)}] \textbf{水平扫描磁场}
    \begin{itemize}
        \item \textbf{作用:} 
        \begin{itemize}
            \item 在垂直于光传播方向上产生静态或低速扫描的磁场,使铷原子基态的磁子能级发生塞曼分裂。
            \item 通过调节磁场强度“扫描”能级间距,为磁共振提供可调的能量参考。
        \end{itemize}
        \item \textbf{要求:}
        \begin{itemize}
            \item 亥姆霍兹线圈需设计为均匀磁场区,保证铷蒸气区域的磁场梯度极小,避免能级展宽。
            \item 若需要磁场线性扫描,驱动电流需由精密控制的斜坡信号或三角波产生。
            \item 需覆盖铷原子塞曼分裂对应的能级间距范围。
        \end{itemize}
    \end{itemize}

    \item[\textbf{(3)}] \textbf{射频磁场}
    \begin{itemize}
        \item \textbf{作用:} 
        \begin{itemize}
            \item 产生交变磁场,当满足共振条件时,诱导铷原子在相邻磁子能级间跃迁(磁共振)。
            \item 通过布居数的重新分布,改变透射光强。
        \end{itemize}
        \item \textbf{要求:}
        \begin{itemize}
            \item 射频频率需与塞曼能级差严格匹配(同步可调)。
            \item 射频磁场方向需与扫描磁场正交,确保有效耦合。
            \item 过高的功率会引起塞曼能级跃迁的功率展宽,需优化至信号清晰可见。
        \end{itemize}
    \end{itemize}
\end{itemize}

\begin{question}
本实验磁共振各发生在 $^{87}$Rb 和 $^{85}$Rb 哪些能级间?
\end{question}

$^{87}$Rb 的磁共振发生在基态的超精细能级 $F = 2$ 的塞曼子能级之间;$^{85}$Rb 的磁共振发生在基态的超精细能级 $F = 3$ 塞曼子能级之间。